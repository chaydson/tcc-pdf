\begin{resumo}[Abstract]
 \begin{otherlanguage*}{english}
    The present work aims to analyze the adoption of DevSecOps practices in the development lifecycle of the open-source software product Brasil Participativo, investigating the impacts on the project's security. 
    Therefore, a case study was conducted, structured by the GQM approach and grounded in a literature review. The method involved integrating static, dynamic, and software composition analysis tools into the continuous integration pipeline
    for the automated collection of quantitative data. The results indicated that automation enabled the effective identification of vulnerabilities from the perspectives of internal and external quality, although the pipeline presented 
    initial low performance indicators due to the high rate of detected failures. It is concluded that the incorporation of DevSecOps practices enabled the early detection of flaws and provided fundamental quantitative parameters for 
    decision-making, reinforcing the importance of continuous monitoring in agile development environments.

    \vspace{\onelineskip}
  
    \noindent 
    \textbf{Key-words}: DevSecOps. Software Security. Case Study. CI/CD. Software Quality.
 \end{otherlanguage*}
\end{resumo}
