\chapter[Conclusão]{Conclusão}

Este capítulo retoma as atividades propostas na seção \ref{sec:cronograma-atividaes} para avaliar sua execução, apresenta os resultados obtidos ao final do trabalho e descreve o planejamento para a fase seguinte da monografia.

\section{Atividades Propostas}

Está apresentado na Tabela \ref{tab:atividades-concluidas-tcc-1} as atividades realizadas a primeira parte da monografia.

\begin{table}[H]
    \centering
	\caption{Situação das Atividades}
    \begin{tabular}{|p{7cm}|p{3cm}|}
        \hline
        \textbf{Atividade} & \textbf{Situação}\\
        \hline
        Contextualização sobre Engenharia de Software Experimental  & Concluída \\
        \hline
        Definição do GQM  & Concluída \\
        \hline
        Elaboração do Protocolo de revisão da literatura  & Concluída \\
        \hline
        Seleção dos artigos  & Concluída \\
        \hline
        Leitura do material selecionado  & Concluída \\
        \hline
        Definição da proposta de solução  & Concluída  \\
        \hline
        Redação da monografia  & Concluída  \\
        \hline
        Revisão  & Concluída  \\
        \hline
        Apresentação  & Agendada \\
        \hline
    \end{tabular} \\[0.5em]
   	Fonte: Autor
	\label{tab:atividades-concluidas-tcc-1}
\end{table}


\section{Resultados Obtidos}

A revisão estruturada da literatura sobre DevSecOps, analisou 40 artigos científicos publicados, predominantemente em revistas, atestando a qualidade dos estudos. Isso permitiu compreender o estado da arte em DevSecOps e responder as perguntas da revisão.

Desse modo, percebeu-se que a integração contínua (CI) e a entrega/implantação contínua (CD), herdadas do DevOps, são cruciais, com validações de build e testes automatizados, para isso são usadas ferramentas SAST, DAST e IAST para garantir a qualidade no desenvolvimento contínuo. Para além de ferramentas, a adoção da cultura Shift Security Left e a Avaliação Contínua de Vulnerabilidades são fundamentais para a segurança dos produtos de software.

Avaliação da segurança de forma quantitativa carece de modelos com validação experimental, mas o uso das métricas de Zhang e a adaptação das métricas DORA para o contexto de segurança das métricas, juntamente com o uso modelos de avaliação de maturidade (OWASP SAMM, OWASP DSOMM e BSIMM) conseguem são maneiras de contornar esse problema.

As vulnerabilidades comumente mitigadas pelas práticas de DevSecOps incluem SQL Injection, Command Injection, XSS, XXE, Buffer Overflow, CSRF, DDoS, MITM, Broken Authentication, Broken Access Control, Security Misconfiguration, Session Hijacking e SSRF

Diversas ferramentas e tecnologias são empregadas no DevSecOps, como Prometheus, Grafana e Loki para Monitoramento; Terraform, Kubernetes e Docker para Infraestrutura; Jenkins, GitLab CI/CD, GitHub Actions, Tekton e ArgoCD para CI/CD; e SonarQube, FindBugs, Snyk, OWASP Dependency-Check, OWASP ZAP, Trivy, Detect Secrets, Asylo, StackHawk, JMeter e Selenium para Testes

\section{Cronograma da Segunda Etapa da Monografia}

O cronograma da segunda etapa da monografia é apresentado na Figura \ref{fig:cronograma-tcc-2}. A primeira etapa é a evolução do trabalho, pois, além de o capítulo referente ao referencial teórico precisa ser melhorado e revisado, também é necessário aplicação das correções apontadas pela banca. 
As próximas etapas são referentes ao estudo de caso, começamdo com a preparação e coleta de dados, seguida análise de dados coletados e documentação dos resultados. Por fim, será realizada a análise dos resultados e redação da monografia, finalizando com a revisão e, posteriormente, a apresentação final do trabalho.

\begin{figure}[h]
    \centering
	\caption{Cronograma TCC 2}
    \includegraphics[width=0.9\textwidth]{figuras/cronograma-tcc-2.png}

    \label{fig:cronograma-tcc-2}
	Fonte: Autor
\end{figure}
