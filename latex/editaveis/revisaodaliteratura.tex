\chapter[Revisão Estruturada da Literatura]{Revisão Estruturada da Literatura}

Este capítulo destina-se a documentar o processo realizado para selecionar o conjunto de obras acadêmicas que compõe a 
bibliografia desta monografia. Para isso, a base de dados Scopus foi escolhida devido à sua característica de indexar 
diversos artigos da área da computação, muitos publicados nos principais meios de divulgação científica (ELSEVIER, 2025).

\section{Protocolo}

O protocolo utilizado para realizar a revisão estruturada da literatura foi baseado no modelo proposto por Kitchenham e Brereton 
(2013). Seu objetivo é tornar possível que outros pesquisadores, partindo do mesmo ponto, cheguem aos mesmos resultados, 
facilitando a replicabilidade em estudos futuros e permitindo a conferência dos resultados obtidos.

\subsection{String de Busca}

A busca em bases de dados acadêmicas requer o uso de um protocolo, pois elas indexam grande quantidade de artigos de várias áreas.
 Seu uso incorreto pode acarretar em um número excessivo de artigos sem relação com o tema da pesquisa ou, inversamente, retornar
  um volume insuficiente de estudos para responder à questão de pesquisa.

Por essa razão, o protocolo PICO foi utilizado para guiar a elaboração de uma string de busca adequada às necessidades da 
monografia. O protocolo, no entanto, precisou ser adaptado, pois sua origem é na medicina e nem todos os seus elementos se 
adequam ao nosso escopo. Uma representação visual que facilita a compreensão do protocolo pode ser vista na 
Figura 4 (PAI et al., 2004).

\begin{figure}[h]
    \centering
	\caption{Abordagem GQM}
    \includegraphics[width=0.7\textwidth]{figuras/pico.png}

    \label{fig:pico}
	Fonte: (PAI et al., 2004)
\end{figure}

Ao adaptar o modelo PICO para o presente trabalho, suas definições adquirem um novo significado no contexto da Engenharia de 
Software. Por exemplo, Patient, outrora usado para indicar o perfil do paciente, passa a representar a área de aplicação, neste 
caso, o desenvolvimento de software. Intervention também sofre adaptação, deixando de significar "tratamento médico" para se 
referir à metodologia avaliada. Não obstantemente, Outcome mantém seu sentido original, referindo-se aos efeitos 
ou consequências observadas. Por fim, Comparison que está relacionada a investigar como a intervenção proposta se relaciona com 
outras propostas de intervenção não pode ser adotada, devido a estar muito mais alinhada com os objetivos da médicina que da engenharia de software.
Assim, a definição de cada um dos elementos usados do modelo estão definidos na Tabela 2.

\begin{table}[h]
    \centering
	\caption{GQM Adaptado}
    \begin{tabular}{|p{4cm}|p{3cm}|p{5cm}|}
        \hline
        \textbf{Elementos} & \textbf{Termo Central} & \textbf{Sinônimos e Termos Relacionados} \\
        \hline
        Population & software development & software systems, online systems, software applications, systems development, application development \\
        \hline
        Intervention & DevSecOps & cybersecurity practices, security automation, secure software development, CI/CD, continuous integration, continuous deployment, continuous delivery, DevOps, security development culture \\
        \hline
        Outcome & security quality & software security, application security, security improvement, security assurance, vulnerability reduction, protection against threats, system security, owasp, cwe, common weakness \\
        \hline
    \end{tabular} \\[0.5em]
   	Fonte: Autor
	\label{tab:pico}
\end{table}

Desta maneira, a string de busca foi construída usando o operador lógico OR entre os termos de cada elemento, com o objetivo de englobar todos os termos da pesquisa.
Já o operador AND foi usado para conectar os diferentes elementos do protocolo PICO, assim restringindo a busca apenas aos estudos que apresentam os termos 
necessários para responder as perguntas de pesquisa. Ademais, foram adicionados os termos própios da base de dados para a realização da consulta, resultando na seguinte string de busca:

\begin{center}
    TITLE-ABS-KEY ( ( "software development" OR "software developments" OR "software system" OR "software systems" OR "online system" OR "online systems" OR "software application" OR "software applications" OR "system development" OR "systems development" OR "application development" OR "applications development" ) AND ( "DevSecOps" OR "cybersecurity practice" OR "cybersecurity practices" OR "security automation" OR "security automations" OR "secure software development" OR "secure software developments" OR "CI/CD" OR "continuous integration" OR "continuous integrations" OR "continuous deployment" OR "continuous deployments" OR "continuous delivery" OR "continuous deliveries" OR "DevOps" OR "security development culture" OR "security development cultures" ) AND ( "security quality" OR "security qualities" OR "software security" OR "software securities" OR "application security" OR "application securities" OR "security improvement" OR "security improvements" OR "security assurance" OR "security assurances" OR "vulnerability reduction" OR "vulnerability reductions" OR "protection against threat" OR "protection against threats" OR "system security" OR "system securities" OR "OWASP" OR "CWE" OR "common weakness" OR "common weaknesses" ) )
\end{center}

\section{Seleção dos Artigos}

Com a string de busca criada, foram estabelecidos os critérios de inclusão e exclusão visando selecionar apenas os artigos relacionados ao contexto da pesquisa. 
Além disso, em primeiro momento, foram lidos o título, resumo e palavras-chave de todos os artigos resultantes da execução da string de busca, isso se deu para
selecionar com maior rigor aqueles estudos que por ventura não correspondessem ao objetivo do trabalho. Posteriormente, os artigos que aprovados pelos critérios de escolha
foram lidos e os dados relevantes para a formulação das respostas das perguntas de pesquisa foram extraídos em um formulário. A Tabela 3 e Tabela 4 contém o protocolo completo.

Ao todo foram analisados 291 artigos resultantes da string de busca, desses 38 foram aceitos e lidos de maneira integral, sendo que esses artigos foram obtidos na base de dados Scopus (ELSEVIER, 2025) no dia 7 de maio de 2025. Afim de complementar os artigos selecionados de forma automatizada
foi realizada uma busca manual com o objetivo de responder de maneira mais acertiva as perguntas de pesquisa, resultando em mais dois artigos, Accelerate State of DevOps 2024 (DORA, 2024) e Quantitative DevSecOps Metrics for Cloud-Based Web Microservices (ZHANG; ZHANG, 2024), usados para compor o referencial teórico, totalizando 40 artigos.
Os artigos selecionados estão dispostos nas Tabela 5, 6, 7 e 8. Para facilitar a compreensão do protocolo de seleção de artigos pode-se verificar a Figura 5, que ilustra todas as etapas explanadas anteriormente.

\begin{table}[h]
    \centering
	\caption{Protocolo de Busca}
    \begin{tabular}{|p{5cm}|p{10cm}|}
        \hline
        Perguntas de Pesquisa &
        \begin{enumerate}[leftmargin=*, label=\arabic*.]
            \item How can the security aspect in the continuous development of web systems be analyzed considering internal and external quality perspectives?
            \item Which DevSecOps practices are most prevalent in organizations?
            \item Which security models are most commonly employed in DevSecOps environments?
            \item Which measures are commonly used for security evaluation?	
            \item How are security practices evaluated?	
            \item In what ways are security measures evaluated in organizations?	
            \item What types of vulnerabilities are mitigated by DevSecOps practices?	
            \item What tools and technologies are used in DevSecOps?
            \item What is the annual volume of DevSecOps publications from 2009 to 2025?	
            \item How many articles were published in academic journals?	
            \item How many studies have experimental validation?	
            \item If the study has experimental validation, what type?	
        \end{enumerate} \\
        \hline
        String de Busca & Tabela 2 \\
        \hline
        Critérios de Inclusão & 
        \begin{enumerate}[leftmargin=*, label=\arabic*.]
            \item Addresses the use of secure development practices
            \item Emphasizes the security quality of web systems
            \item Evaluates quality models with a focus on security
            \item Focuses on software products
            \item Includes experimental validation
        \end{enumerate} \\
        \hline
    \end{tabular} \\[0.5em]
   	Fonte: Autor
	\label{tab:protocolo}
\end{table}

\begin{table}[h]
    \centering
	\caption{Continuação do Protocolo de Busca}
    \begin{tabular}{|p{5cm}|p{10cm}|}
        \hline
        Perguntas de Exclusão &
        \begin{enumerate}[leftmargin=*, label=\arabic*.]
            \item Articles in languages other than English or Portuguese
            \item Duplicate publication
            \item Publications with a release date prior to 2009
            \item Studies focusing on hardware, mobile, IoT security, or other topics unrelated to web systems.
        \end{enumerate} \\
        \hline
        Formulário de Extração & 
        \begin{enumerate}[leftmargin=*, label=\arabic*.]
            \item Title
            \item Abstract
            \item Publication Year	
            \item Publication Source	
            \item Authors
            \item Keywords
            \item Prevalent DevSecOps Practices	
            \item Security Models Employed	
            \item Security Evaluation Measures	
            \item Security Practices Evaluation	
            \item Security Models Analysis	
            \item Organizational Security Evaluation	
            \item Mitigated Vulnerabilities	
            \item Tools and Technologies
            \item Published in Academic Journal	
            \item Experimental Validation	
            \item Type of Experimental Validation	
            \item Secondary Research	
        \end{enumerate} \\
        \hline
    \end{tabular} \\[0.5em]
   	Fonte: Autor
	\label{tab:protocolo-continuacao}
\end{table}

\begin{table}[h]
    \centering
    \caption{Artigos Selecionados}
    \begin{tabular}{|p{1cm}|p{8cm}|p{3.5cm}|p{3cm}|}
        \hline
        \textbf{N°} & \textbf{Título} & \textbf{Publicado em Revista} & \textbf{Validação Experimental} \\
        \hline
        1 & Development of Secure Software Based on the New DevSecOps Technology & Sim & Não \\
        \hline
        2 & Automating Security in a Continuous Integration Pipeline & Não & Não \\
        \hline
        3 & Extensive Review of Threat Models for DevSecOps & Sim & Não \\
        \hline
        4 & Implementing and Automating Security Scanning to a DevSecOps CI/CD Pipeline & Sim & Não \\
        \hline
        5 & Automating Static Code Analysis Through CI/CD Pipeline Integration & Sim & Sim \\
        \hline
        6 & Design and Practice of Security Architecture via DevSecOps Technology & Sim & Sim \\
        \hline
        7 & Implementation of DevSecOps by Integrating Static and Dynamic Security Testing in CI/CD Pipelines & Sim & Não \\
        \hline
        8 & Research of Static Application Security Testing Technique Problems and Methods for Solving Them & Sim & Não \\
        \hline
        9 & A Large-scale Fine-grained Empirical Study on Security Concerns in Open-source Software & Sim & Sim \\
        \hline
        10 & Evolution of secure development lifecycles and maturity models in the context of hosted solutions & Não & Não \\
        \hline
    \end{tabular} \\[0.5em]
    \label{tab:artigos-selecionados}
    \small Fonte: Autor
\end{table}

\begin{table}[h]
    \centering
    \caption{Continuação dos Artigos Selecionados}
    \begin{tabular}{|p{1cm}|p{8cm}|p{3.5cm}|p{3cm}|}
        \hline
        \textbf{N°} & \textbf{Título} & \textbf{Publicado em Revista} & \textbf{Validação Experimental} \\
        \hline
        11 & Automation and DevSecOps: Streamlining Security Measures in Financial System & Sim & Não \\
        \hline
        12 & Securing the development and delivery of modern applications & Sim & Não \\
        \hline
        13 & You cannot improve what you do not measure: A triangulation study of software security metrics & Sim & Sim \\
        \hline
        14 & On DevSecOps and Risk Management in Critical Infrastructures: Practitioners' Insights on Needs and Goals & Sim & Sim \\
        \hline
        15 & Container Security in Cloud Environments: A Comprehensive Analysis and Future Directions for DevSecOps & Não & Sim \\
        \hline
        16 & Microservices-based DevSecOps Platform using Pipeline and Open Source Software & Não & Não \\
        \hline
        17 & Securing the Digital Frontier: A Proactive Approach to Software Development & Sim & Não \\
        \hline
        18 & A Secure Software Development Methodology for Enterprise Business Applications & Sim & Sim \\
        \hline
        19 & Building Resilient CICD Pipelines: A DevOps Security-First Framework & Sim & Não \\
        \hline
        20 & Review of Techniques for Integrating Security in Software Development Lifecycle & Não & Não \\
        \hline
    \end{tabular} \\[0.5em]
    \label{tab:artigos-selecionados-2}
    \small Fonte: Autor
\end{table}

\begin{table}[h]
    \centering
    \caption{Continuação dos Artigos Selecionados (cont.)}
    \begin{tabular}{|p{1cm}|p{8cm}|p{3.5cm}|p{3cm}|}
        \hline
        \textbf{N°} & \textbf{Título} & \textbf{Publicado em Revista} & \textbf{Validação Experimental} \\
        \hline
        21 & A hierarchical model for quantifying software security based on static analysis alerts and software metrics & Sim & Sim \\
        \hline
        22 & A preventive secure software development model for a software factory: A case study & Sim & Sim \\
        \hline
        23 & Security impacts of sub-optimal DevSecOps implementations in a highly regulated environment & Sim & Sim \\
        \hline
        24 & A survey and comparison of secure software development standards & Sim & Sim \\
        \hline
        25 & Continuous Security Testing: A Case Study on Integrating Dynamic Security Testing Tools in CI/CD Pipelines & Sim & Sim \\
        \hline
        26 & Infiltrating Security into Development: Exploring the World's Largest Software Security Study & Não & Sim \\
        \hline
        27 & Challenges and solutions when adopting DevSecOps: A systematic review & Sim & Não \\
        \hline
        28 & Systematic Mapping Study on Security Approaches in Secure Software Engineering & Sim & Não \\
        \hline
        29 & Systematic Literature Review on Security Risks and its Practices in Secure Software Development & Sim & Não \\
        \hline
        30 & BP: Security concerns and best practices for automation of software deployment processes: An industrial case study & Sim & Sim \\
        \hline
    \end{tabular} \\[0.5em]
    \label{tab:artigos-selecionados-3}
    \small Fonte: Autor
\end{table}

\begin{table}[h]
    \centering
    \caption{Continuação dos Artigos Selecionados (cont.)}
    \begin{tabular}{|p{1cm}|p{8cm}|p{3.5cm}|p{3cm}|}
        \hline
        \textbf{N°} & \textbf{Título} & \textbf{Publicado em Revista} & \textbf{Validação Experimental} \\
        \hline
        31 & Static analysis for web service security - Tools \& techniques for a secure development life cycle & Sim & Não \\
        \hline
        32 & Security characterization for evaluation of software architectures using ATAM & Sim & Sim \\
        \hline
        33 & Software security & Sim & Não \\
        \hline
        34 & Using the ISO/IEC 27034 as reference to develop an application security control library & Sim & Sim \\
        \hline
        35 & Hunting for aardvarks: Can software security be measured? & Não & Não \\
        \hline
        36 & Francois Raynaud on DevSecOps & Sim & Não \\
        \hline
        37 & Integrating application security into software development & Sim & Não \\
        \hline
        38 & Busting a myth: Review of agile security engineering methods & Sim & Não \\
        \hline
    \end{tabular} \\[0.5em]
    \label{tab:artigos-selecionados-4}
    \small Fonte: Autor
\end{table}

\begin{figure}[h]
    \centering
	\caption{Seleção dos Artigos}
    \includegraphics[width=0.7\textwidth]{figuras/fluxograma-artigos.png}

    \label{fig:fluxograma-artigos}
	Fonte: Autor
\end{figure}

\section{Resultados}

Esta seção apresentará os resultados obtidos com a leitura do material coletado. Durante o processo de leitura, com o objetivo de facilitar a elaboração das respostas da perguntas de pesquisa
e para possibilitar a aferição da revisão da literatura, foi construída uma planilha 
\footnote{Planilha com o resultado da revisão: \url{https://docs.google.com/spreadsheets/d/1HdgzzRaP8YlS_08hZhIKaUKV9HKP2dTaq4xWVKFzh84/edit?usp=sharing}} contendo todos os dados extraídos dos artigos, a planilha está separada em vários páginas, 
pois cada página está relacionada com uma pergunta de pesquisa ou um conjunto de perguntas de pesquisa semelhantes, desse modo, possibilitando que o leitor consiga enxergar como cada artigo responde as perguntas de pesquisa.
Assim, com os dados extraídos dos artigos foi possível estruturar o conhecimento para responder as perguntas de pesquisa baseado no estado da arte sobre o tema.

Nas subseções a seguir são apresentadas as respostas para as perguntas de pesquisa.

\subsection{Which DevSecOps practices are most prevalent in organizations?}
\begin{itemize}
    \item Metodologias e Cultura: Shift Security Left, Continuous Integration, Continuous Delivery, Continuous Deployment, Continuous Feedback, Continuous Vulnerability Assessment
    \item Testes de Segurança: SAST, DAST, IAST, Fuzz testing, BDST, WAST, SAS
    \item Infraestrutura e Operações: IaC, Secure Infrastructure, Container Security, 
    \item Acompanhamento: Automation, Monitoring, Logging
\end{itemize}



\subsection{Which security models are most commonly employed in DevSecOps environments?}
\begin{itemize}
    \item OWASP SAMM
    \item OWASP DSOMM
    \item BSIMM
    \item DORA
\end{itemize}


\subsection{Which measures are commonly used for security evaluation?}
\begin{itemize}
    \item Non-Comment Lines of Code
    \item Design Defect Ratio
    \item Shared or Unknown Library Ratio
    \item Technical Debt Ratio
    \item Continuous Deployment Cycles Score
    \item Mean Change Lead Time
    \item Mean Time to Recover
    \item Mean Number of Test Cases Per Parameter
    \item Points of Environmental Risk
    \item Time for Response
    \item Throughput
    \item Errors Per Time Unit
    \item Change lead time
    \item Deployment frequency
    \item Change fail percentage
    \item Failed deployment recovery time
\end{itemize}



\subsection{How are security practices evaluated?}
\begin{itemize}
    \item auditorias
    \item avaliação da maturidade
    \item avaliação de métricas
\end{itemize}



\subsection{In what ways are security measures evaluated in organizations?}
\begin{itemize}
    \item Impacto das métricas nos KPIs
    \item Dashboards e relatórios automatizados
    \item Adaptação ao compliance/regulação
\end{itemize}


\subsection{What types of vulnerabilities are mitigated by DevSecOps practices?	}
\begin{table}[h]
    \centering
    \caption{Vulnerabilidades Reduzidas}
    \begin{tabular}{|p{5cm}|p{6cm}|p{4cm}|}
        \hline
        \textbf{Vulnerabilidade} & \textbf{Descrição} & \textbf{Referência} \\
        \hline
        SQL Injection & - & - \\
        \hline
        Command Injection & - & - \\
        \hline
        XSS & - & - \\
        \hline
        XXE & - & - \\
        \hline
        Buffer Overflow & - & - \\
        \hline
        CSRF & - & - \\
        \hline
        DDoS & - & - \\
        \hline
        MITM & - & - \\
        \hline
        Broken Authentication & - & - \\
        \hline
        Broken Access Control & - & - \\
        \hline
        Security Misconfiguration & - & - \\
        \hline
        Session Hijacking & - & - \\
        \hline
        SSRF & - & - \\
        \hline
    \end{tabular} \\[0.5em]
    \label{tab:vulnerabilidades}
    \small Fonte: Autor
\end{table}



\subsection{What tools and technologies are used in DevSecOps?}
\begin{itemize}
    \item Monitoramento: Prometheus, Grafana, Loki
    \item Infraestrutura: Terraform, Kubernetes, Docker
    \item CI/CD: Jenkins, GitLab CI/CD, GitHub Actions, Tekton, ArgoCD
    \item Testes: SonarQube, FindBugs, Snyk, OWASP Dependency-Check, OWASP ZAP, Trivy, Detect Secrets, Asylo, StackHawk, JMeter, Selenium
\end{itemize}



\subsection{What is the annual volume of DevSecOps publications from 2009 to 2025?}
\begin{figure}[h]
    \centering
	\caption{Volume Anual de Artigos entre 2009 e 2025}
    \includegraphics[width=0.9\textwidth]{figuras/volume-anual-artigos.png}

    \label{fig:volume-anual}
	Fonte: Autor
\end{figure}



\subsection{How many articles were published in academic journals?}
\begin{figure}[h]
    \centering
	\caption{Artigos Publicados em Revistas}
    \includegraphics[width=1.0\textwidth]{figuras/artigos-publicados.png}

    \label{fig:artigos-publicados}
	Fonte: Autor
\end{figure}



\subsection{How many studies have experimental validation?}
\begin{figure}[h]
    \centering
	\caption{Quantidade de Artigos com Validação Experimental}
    \includegraphics[width=1.0\textwidth]{figuras/quantidade-experimental.png}

    \label{fig:quantidade-experimental}
	Fonte: Autor
\end{figure}



\subsection{If the study has experimental validation, what type?}
\begin{figure}[h]
    \centering
	\caption{Tipos de Validação Experimental dos Artigos}
    \includegraphics[width=1.0\textwidth]{figuras/tipos-validacao.png}

    \label{fig:tipos-validacao}
	Fonte: Autor
\end{figure}