\chapter[Revisão Estruturada da Literatura]{Revisão Estruturada da Literatura}

Este capítulo destina-se a documentar o processo realizado para selecionar o conjunto de obras acadêmicas que compõe a 
bibliografia desta monografia. Para isso, a base de dados Scopus foi escolhida devido à sua característica de indexar 
diversos artigos da área da computação, muitos publicados nos principais meios de divulgação científica \cite{Scopus2025}.

\section{Protocolo}

O protocolo utilizado para realizar a revisão estruturada da literatura foi baseado no modelo proposto por \citeonline{KITCHENHAM20132049}. Seu objetivo é tornar possível que outros pesquisadores, partindo do mesmo ponto, cheguem aos mesmos resultados, 
facilitando a replicabilidade em estudos futuros e permitindo a conferência dos resultados obtidos.

\subsection{String de Busca}

A busca em bases de dados acadêmicas requer o uso de um protocolo, pois elas indexam grande quantidade de artigos de várias áreas.
Seu uso incorreto pode acarretar em um número excessivo de artigos sem relação com o tema da pesquisa ou, inversamente, retornar
um volume insuficiente de estudos para responder à questão de pesquisa.

Por essa razão, o protocolo PICO foi utilizado para guiar a elaboração de uma string de busca adequada às necessidades da 
monografia. O protocolo, no entanto, precisou ser adaptado, pois sua origem é na medicina e nem todos os seus elementos se 
adequam ao nosso escopo. Uma representação visual que facilita a compreensão do protocolo pode ser vista na Figura 4 \cite{Pai2004}.

\begin{figure}[h]
    \centering
	\caption{Abordagem GQM}
    \includegraphics[width=0.7\textwidth]{figuras/pico.png}

    \label{fig:pico}
	Fonte: \citeonline{Pai2004}
\end{figure}

Ao adaptar o modelo PICO para o presente trabalho, suas definições adquirem um novo significado no contexto da Engenharia de 
Software. Por exemplo, Patient, outrora usado para indicar o perfil do paciente, passa a representar a área de aplicação, neste 
caso, o desenvolvimento de software. Intervention também sofre adaptação, deixando de significar "tratamento médico" para se 
referir à metodologia avaliada. Não obstantemente, Outcome mantém seu sentido original, referindo-se aos efeitos 
ou consequências observadas. Por fim, Comparison que está relacionada a investigar como a intervenção proposta se relaciona com 
outras propostas de intervenção não pode ser adotada, devido a estar muito mais alinhada com os objetivos da médicina que da engenharia de software.
Assim, a definição de cada um dos elementos usados do modelo estão definidos na Tabela 2.

\begin{table}[h]
    \centering
	\caption{GQM Adaptado}
    \begin{tabular}{|p{4cm}|p{3cm}|p{5cm}|}
        \hline
        \textbf{Elementos} & \textbf{Termo Central} & \textbf{Sinônimos e Termos Relacionados} \\
        \hline
        Population & software development & software systems, online systems, software applications, systems development, application development \\
        \hline
        Intervention & DevSecOps & cybersecurity practices, security automation, secure software development, CI/CD, continuous integration, continuous deployment, continuous delivery, DevOps, security development culture \\
        \hline
        Outcome & security quality & software security, application security, security improvement, security assurance, vulnerability reduction, protection against threats, system security, owasp, cwe, common weakness \\
        \hline
    \end{tabular} \\[0.5em]
   	Fonte: Autor
	\label{tab:pico}
\end{table}

Desta maneira, a string de busca foi construída usando o operador lógico OR entre os termos de cada elemento, com o objetivo de englobar todos os termos da pesquisa.
Já o operador AND foi usado para conectar os diferentes elementos do protocolo PICO, assim restringindo a busca apenas aos estudos que apresentam os termos 
necessários para responder as perguntas de pesquisa. Ademais, foram adicionados os termos própios da base de dados para a realização da consulta, resultando na seguinte string de busca:

\begin{center}
    TITLE-ABS-KEY ( ( "software development" OR "software developments" OR "software system" OR "software systems" OR "online system" OR "online systems" OR "software application" OR "software applications" OR "system development" OR "systems development" OR "application development" OR "applications development" ) AND ( "DevSecOps" OR "cybersecurity practice" OR "cybersecurity practices" OR "security automation" OR "security automations" OR "secure software development" OR "secure software developments" OR "CI/CD" OR "continuous integration" OR "continuous integrations" OR "continuous deployment" OR "continuous deployments" OR "continuous delivery" OR "continuous deliveries" OR "DevOps" OR "security development culture" OR "security development cultures" ) AND ( "security quality" OR "security qualities" OR "software security" OR "software securities" OR "application security" OR "application securities" OR "security improvement" OR "security improvements" OR "security assurance" OR "security assurances" OR "vulnerability reduction" OR "vulnerability reductions" OR "protection against threat" OR "protection against threats" OR "system security" OR "system securities" OR "OWASP" OR "CWE" OR "common weakness" OR "common weaknesses" ) )
\end{center}

\section{Seleção dos Artigos}

Com a string de busca criada, foram estabelecidos os critérios de inclusão e exclusão visando selecionar apenas os artigos relacionados ao contexto da pesquisa. 
Além disso, em primeiro momento, foram lidos o título, resumo e palavras-chave de todos os artigos resultantes da execução da string de busca, isso se deu para
selecionar com maior rigor aqueles estudos que por ventura não correspondessem ao objetivo do trabalho. Posteriormente, os artigos que aprovados pelos critérios de escolha
foram lidos e os dados relevantes para a formulação das respostas das perguntas de pesquisa foram extraídos em um formulário. A Tabela 3 e Tabela 4 contém o protocolo completo.

Ao todo foram analisados 291 artigos resultantes da string de busca, desses 38 foram aceitos e lidos de maneira integral, sendo que esses artigos foram obtidos na base de dados \citeonline{Scopus2025} no dia 7 de maio de 2025. Afim de complementar os artigos selecionados de forma automatizada
foi realizada uma busca manual com o objetivo de responder de maneira mais acertiva as perguntas de pesquisa, resultando em mais dois artigos, Accelerate State of DevOps 2024 (DORA, 2024) e Quantitative DevSecOps Metrics for Cloud-Based Web Microservices \cite{Zhang2024160317}, usados para compor o referencial teórico, totalizando 40 artigos.
Os artigos selecionados estão dispostos nas Tabela 5, 6, 7 e 8. Para facilitar a compreensão do protocolo de seleção de artigos pode-se verificar a Figura 5, que ilustra todas as etapas explanadas anteriormente.

\begin{table}[h]
    \centering
	\caption{Protocolo de Busca}
    \begin{tabular}{|p{5cm}|p{10cm}|}
        \hline
        Perguntas de Pesquisa &
        \begin{enumerate}[leftmargin=*, label=\arabic*.]
            \item How can the security aspect in the continuous development of web systems be analyzed considering internal and external quality perspectives?
            \item Which DevSecOps practices are most prevalent in organizations?
            \item Which security models are most commonly employed in DevSecOps environments?
            \item Which measures are commonly used for security evaluation?	
            \item How are security practices evaluated?	
            \item In what ways are security measures evaluated in organizations?	
            \item What types of vulnerabilities are mitigated by DevSecOps practices?	
            \item What tools and technologies are used in DevSecOps?
            \item What is the annual volume of DevSecOps publications from 2009 to 2025?	
            \item How many articles were published in academic journals?	
            \item How many studies have experimental validation?	
            \item If the study has experimental validation, what type?	
        \end{enumerate} \\
        \hline
        String de Busca & Tabela 2 \\
        \hline
        Critérios de Inclusão & 
        \begin{enumerate}[leftmargin=*, label=\arabic*.]
            \item Addresses the use of secure development practices
            \item Emphasizes the security quality of web systems
            \item Evaluates quality models with a focus on security
            \item Focuses on software products
            \item Includes experimental validation
        \end{enumerate} \\
        \hline
    \end{tabular} \\[0.5em]
   	Fonte: Autor
	\label{tab:protocolo}
\end{table}

\begin{table}[h]
    \centering
	\caption{Continuação do Protocolo de Busca}
    \begin{tabular}{|p{5cm}|p{10cm}|}
        \hline
        Perguntas de Exclusão &
        \begin{enumerate}[leftmargin=*, label=\arabic*.]
            \item Articles in languages other than English or Portuguese
            \item Duplicate publication
            \item Publications with a release date prior to 2009
            \item Studies focusing on hardware, mobile, IoT security, or other topics unrelated to web systems.
        \end{enumerate} \\
        \hline
        Formulário de Extração & 
        \begin{enumerate}[leftmargin=*, label=\arabic*.]
            \item Title
            \item Abstract
            \item Publication Year	
            \item Publication Source	
            \item Authors
            \item Keywords
            \item Prevalent DevSecOps Practices	
            \item Security Models Employed	
            \item Security Evaluation Measures	
            \item Security Practices Evaluation	
            \item Security Models Analysis	
            \item Organizational Security Evaluation	
            \item Mitigated Vulnerabilities	
            \item Tools and Technologies
            \item Published in Academic Journal	
            \item Experimental Validation	
            \item Type of Experimental Validation	
            \item Secondary Research	
        \end{enumerate} \\
        \hline
    \end{tabular} \\[0.5em]
   	Fonte: Autor
	\label{tab:protocolo-continuacao}
\end{table}

\begin{table}[h]
    \centering
    \caption{Artigos Selecionados}
    \begin{tabular}{|p{1cm}|p{8cm}|p{3.5cm}|p{3cm}|}
        \hline
        \textbf{N°} & \textbf{Título} & \textbf{Publicado em Revista} & \textbf{Validação Experimental} \\
        \hline
        1 & Development of Secure Software Based on the New DevSecOps Technology & Sim & Não \\
        \hline
        2 & Automating Security in a Continuous Integration Pipeline & Não & Não \\
        \hline
        3 & Extensive Review of Threat Models for DevSecOps & Sim & Não \\
        \hline
        4 & Implementing and Automating Security Scanning to a DevSecOps CI/CD Pipeline & Sim & Não \\
        \hline
        5 & Automating Static Code Analysis Through CI/CD Pipeline Integration & Sim & Sim \\
        \hline
        6 & Design and Practice of Security Architecture via DevSecOps Technology & Sim & Sim \\
        \hline
        7 & Implementation of DevSecOps by Integrating Static and Dynamic Security Testing in CI/CD Pipelines & Sim & Não \\
        \hline
        8 & Research of Static Application Security Testing Technique Problems and Methods for Solving Them & Sim & Não \\
        \hline
        9 & A Large-scale Fine-grained Empirical Study on Security Concerns in Open-source Software & Sim & Sim \\
        \hline
        10 & Evolution of secure development lifecycles and maturity models in the context of hosted solutions & Não & Não \\
        \hline
    \end{tabular} \\[0.5em]
    \label{tab:artigos-selecionados}
    \small Fonte: Autor
\end{table}

\begin{table}[h]
    \centering
    \caption{Continuação dos Artigos Selecionados}
    \begin{tabular}{|p{1cm}|p{8cm}|p{3.5cm}|p{3cm}|}
        \hline
        \textbf{N°} & \textbf{Título} & \textbf{Publicado em Revista} & \textbf{Validação Experimental} \\
        \hline
        11 & Automation and DevSecOps: Streamlining Security Measures in Financial System & Sim & Não \\
        \hline
        12 & Securing the development and delivery of modern applications & Sim & Não \\
        \hline
        13 & You cannot improve what you do not measure: A triangulation study of software security metrics & Sim & Sim \\
        \hline
        14 & On DevSecOps and Risk Management in Critical Infrastructures: Practitioners' Insights on Needs and Goals & Sim & Sim \\
        \hline
        15 & Container Security in Cloud Environments: A Comprehensive Analysis and Future Directions for DevSecOps & Não & Sim \\
        \hline
        16 & Microservices-based DevSecOps Platform using Pipeline and Open Source Software & Não & Não \\
        \hline
        17 & Securing the Digital Frontier: A Proactive Approach to Software Development & Sim & Não \\
        \hline
        18 & A Secure Software Development Methodology for Enterprise Business Applications & Sim & Sim \\
        \hline
        19 & Building Resilient CICD Pipelines: A DevOps Security-First Framework & Sim & Não \\
        \hline
        20 & Review of Techniques for Integrating Security in Software Development Lifecycle & Não & Não \\
        \hline
    \end{tabular} \\[0.5em]
    \label{tab:artigos-selecionados-2}
    \small Fonte: Autor
\end{table}

\begin{table}[h]
    \centering
    \caption{Continuação dos Artigos Selecionados (cont.)}
    \begin{tabular}{|p{1cm}|p{8cm}|p{3.5cm}|p{3cm}|}
        \hline
        \textbf{N°} & \textbf{Título} & \textbf{Publicado em Revista} & \textbf{Validação Experimental} \\
        \hline
        21 & A hierarchical model for quantifying software security based on static analysis alerts and software metrics & Sim & Sim \\
        \hline
        22 & A preventive secure software development model for a software factory: A case study & Sim & Sim \\
        \hline
        23 & Security impacts of sub-optimal DevSecOps implementations in a highly regulated environment & Sim & Sim \\
        \hline
        24 & A survey and comparison of secure software development standards & Sim & Sim \\
        \hline
        25 & Continuous Security Testing: A Case Study on Integrating Dynamic Security Testing Tools in CI/CD Pipelines & Sim & Sim \\
        \hline
        26 & Infiltrating Security into Development: Exploring the World's Largest Software Security Study & Não & Sim \\
        \hline
        27 & Challenges and solutions when adopting DevSecOps: A systematic review & Sim & Não \\
        \hline
        28 & Systematic Mapping Study on Security Approaches in Secure Software Engineering & Sim & Não \\
        \hline
        29 & Systematic Literature Review on Security Risks and its Practices in Secure Software Development & Sim & Não \\
        \hline
        30 & BP: Security concerns and best practices for automation of software deployment processes: An industrial case study & Sim & Sim \\
        \hline
    \end{tabular} \\[0.5em]
    \label{tab:artigos-selecionados-3}
    \small Fonte: Autor
\end{table}

\begin{table}[h]
    \centering
    \caption{Continuação dos Artigos Selecionados (cont.)}
    \begin{tabular}{|p{1cm}|p{8cm}|p{3.5cm}|p{3cm}|}
        \hline
        \textbf{N°} & \textbf{Título} & \textbf{Publicado em Revista} & \textbf{Validação Experimental} \\
        \hline
        31 & Static analysis for web service security - Tools \& techniques for a secure development life cycle & Sim & Não \\
        \hline
        32 & Security characterization for evaluation of software architectures using ATAM & Sim & Sim \\
        \hline
        33 & Software security & Sim & Não \\
        \hline
        34 & Using the ISO/IEC 27034 as reference to develop an application security control library & Sim & Sim \\
        \hline
        35 & Hunting for aardvarks: Can software security be measured? & Não & Não \\
        \hline
        36 & Francois Raynaud on DevSecOps & Sim & Não \\
        \hline
        37 & Integrating application security into software development & Sim & Não \\
        \hline
        38 & Busting a myth: Review of agile security engineering methods & Sim & Não \\
        \hline
    \end{tabular} \\[0.5em]
    \label{tab:artigos-selecionados-4}
    \small Fonte: Autor
\end{table}

\begin{figure}[h]
    \centering
	\caption{Seleção dos Artigos}
    \includegraphics[width=0.7\textwidth]{figuras/fluxograma-artigos.png}

    \label{fig:fluxograma-artigos}
	Fonte: Autor
\end{figure}

\section{Resultados}

Esta seção apresentará os resultados obtidos com a leitura do material coletado. Durante o processo de leitura, com o objetivo de facilitar a elaboração das respostas da perguntas de pesquisa
e para possibilitar a aferição da revisão da literatura, foi construída uma planilha 
\footnote{Planilha com o resultado da revisão: \url{https://docs.google.com/spreadsheets/d/1HdgzzRaP8YlS_08hZhIKaUKV9HKP2dTaq4xWVKFzh84/edit?usp=sharing}} contendo todos os dados extraídos dos artigos, a planilha está separada em várias páginas, 
pois cada página está relacionada com uma pergunta de pesquisa ou um conjunto de perguntas de pesquisa semelhantes, desse modo, possibilitando que o leitor consiga enxergar como cada artigo responde as perguntas de pesquisa.
Assim, com os dados extraídos dos artigos foi possível estruturar o conhecimento para responder as perguntas de pesquisa baseado no estado da arte sobre o tema.

Nas subseções a seguir são apresentadas as respostas para as perguntas de pesquisa.

\subsection{Which DevSecOps practices are most prevalent in organizations?}
\begin{itemize}
    \item Metodologias e Cultura
    \begin{itemize}
        \item Shift Security Left: É o princípio fundamental do DevSecOps. Ele consiste em deslocar as práticas de segurança para o início do processo de desenvolvimento do projeto, em vez de deixá-las para o final, como acontece na maioria dos casos. Essa postura eleva a prioridade da segurança no projeto, permitindo que as vulnerabilidades sejam tratadas mais cedo \cite{Rajapakse2022}.
        \item Continuos Vulnerability Assessment: Juntamente com Shift Left, a Avaliação Contínua de Vulnerabilidades forma os pilares do DevSecOps. Nessa prática a segurança do software é continuamente verificada, não somente durante o desenvolvimento, mas também após a implantação do software, pois é necessário monitorar a segurança do sistema durante todo o seu ciclo de vida \cite{Rajapakse2022}.
        \item CI/CD: Continuous Integration (CI) e Continuous Delivery/Deployment (CD) integram o DevSecOps, que se trata de uma evolução do DevOps. Como esses conceitos são herdados, é importante defini-los para o completo entendimento da metodologia. CI se refere à prática de realizar continuamente integrações de código na branch principal do código. Como essa atividade envolve a alteração da ramificação principal, ela é validada por meio de verificadores de build e testes automatizados. Já o CD visa à implantação automática e contínua das atualizações de código no ambiente de produção. Para isso, o novo código passa por verificações de qualidade e, se tudo estiver de acordo com a política estabelecida, é publicado sem intervenção humana por meio do pipeline \cite{Rajapakse2022}.
        \item Continuous Feedback: Obter feedback de maneira contínua e rápida é vital em ambientes de entrega contínua. Nessa abordagem, os problemas são rapidamente identificados, possibilitando que as informações cheguem depressa às equipes para que as medidas necessárias sejam tomadas. Os métodos tradicionais de coleta de dados e feedback são lentos demais para a agilidade requisitada em ambientes DevSecOps, e essa lentidão afeta diretamente a velocidade de localização e resolução dos problemas \cite{Rajapakse2022}.
    \end{itemize}
    \item Testes de Segurança
    \begin{itemize}
        \item SAST, DAST, IAST: Ferramentas SAST executam testes estáticos, ou seja, são realizados apenas sobre o código-fonte, podendo indicar potenciais vulnerabilidades e más práticas, conhecidas como code smells. Por outro lado, as ferramentas DAST analisam o software em execução, o que possibilita testar o comportamento do sistema em uso. As ferramentas IAST, por sua vez, fornecem uma abordagem híbrida, incorporando características da análise estática e dinâmica; são modernas e possuem boa integração com ambientes de desenvolvimento contínuo \cite{Rajapakse2022}. 
        \item Fuzz Testing: O teste de fuzzing, como também é conhecido, destaca-se entre as práticas de teste de segurança. Ele consiste em fornecer à aplicação entradas aleatórias, inválidas e inesperadas, de forma a verificar possíveis casos de borda não testados \cite{Masood2015}.
        \item BDST: Os testes BDST são baseados no BDD, porém aplicados ao contexto de testes de segurança. Como os testes descrevem seu comportamento em linguagem natural, pessoas de fora da área de desenvolvimento e segurança de software conseguem compreender os testes realizados \cite{Rangnau2020145}.
    \end{itemize}
    \item Infraestrutura
    \begin{itemize}
        \item IaC: Essa prática consiste em definir a infraestrutura do projeto como código. É altamente usada em contextos DevSecOps, pois possibilita a rápida configuração da infraestrutura. Como está definida em código, pode-se realizar seu versionamento, teste e implantação de forma ágil e adequada a ambientes complexos que necessitam de segurança robusta \cite{Rajapakse2022}.
    \end{itemize}
    \item Acompanhamento
    \begin{itemize}
        \item Monitoring e Logging: Muitas vezes, o registro e a documentação dos eventos relacionados à segurança são negligenciados pelas equipes, o que consiste em um grande erro. Registrar e monitorar fornece um feedback valioso que pode ser utilizado para tomar decisões estratégicas ou realizar auditorias \cite{Rajapakse2022}.
    \end{itemize}
\end{itemize}



\subsection{Which security models are most commonly employed in DevSecOps environments?}
\begin{itemize}
    \item OWASP SAMM: O Software Assurance Maturity Model é um modelo prescritivo de maturidade de segurança de software, ou seja, ele informa quais atividades precisam ser realizadas, diferentemente de um modelo descritivo, que apenas descreve as atividades. Sua estrutura é adequada para diferentes tipos e tamanhos de empresas, bem como para distintas metodologias de desenvolvimento, como cascata e ágil \cite{Lange2024}.
    \item OWASP DSOMM: Embora baseado no SAMM, o DevSecOps Maturity Model nasce devido à necessidade de um modelo adequado para ambientes DevOps, onde a segurança é parte essencial do ciclo de vida. Também é prescritivo, mas, diferentemente do SAMM, suas atividades são definidas em um nível mais próximo do programador do que da gestão. Desse modo, ele fornece com detalhes técnicos os requisitos necessários para atingir cada um dos níveis de maturidade em suas diferentes dimensões \cite{Lange2024}.
    \item BSIMM: Diferentemente dos outros, este é um modelo descritivo, ou seja, ele descreve atividades sem exigir que sejam implementadas. Outra diferença fundamental é que se trata de um modelo proprietário, mantido pela Synopsys. A aplicação do modelo, bem como sua metodologia de avaliação, só estão disponíveis mediante contratação dos serviços da empresa \cite{Lange2024}.
\end{itemize}


\subsection{Which measures are commonly used for security evaluation?}
\begin{itemize}
    \item Métricas de Zhang: Segundo \citeonline{Zhang2024160317}, medir de forma eficaz as características de softwares web é fundamental, porém pouco explorado. Para resolver este problema, eles realizaram uma revisão sistemática da literatura e definiram doze métricas voltadas a atender às necessidades dos sistemas web que usam DevSecOps. Elas permitem quantificar o desempenho do serviço, a segurança e a eficiência da operação, apoiando, assim, as tomadas de decisão e a melhoria contínua das práticas.  
    \begin{itemize}
        \item Non-Comment Lines of Code: Tamanho do código-fonte, excluindo comentários e linhas em branco.
        \item Design Defect Ratio: Proporção de defeitos de design em relação às linhas de código não comentadas.
        \item Shared or Unknown Library Ratio: Proporção de bibliotecas compartilhadas ou não verificadas em um serviço.
        \item Technical Debt Ratio: Compara o custo de resolução de um débito técnico com o custo total do código-fonte.
        \item Continuous Deployment Cycles Score: Pontuação dos ciclos de implantação contínua.
        \item Mean Change Lead Time: Tempo médio que uma mudança leva desde o commit até chegar à produção.
        \item Mean Time to Recover: Tempo médio para recuperação de falhas causadas nos pipelines de CI/CD.
        \item Mean Number of Test Cases Per Parameter: Média de casos de teste por parâmetro.
        \item Points of Environmental Risk: Total de riscos de segurança não resolvidos em produção.
        \item Time for Response: Tempo médio que as equipes de desenvolvimento levam para solucionar incidentes de segurança.
        \item Throughput: Mede a capacidade de processamento de um serviço.
        \item Errors Per Time Unit: Taxa de erros em determinada unidade de tempo.
    \end{itemize}   
    \item Métricas DORA: O DevOps Research and Assessment (DORA) é um dos principais programas de pesquisa do mundo na área de DevOps. Esse programa, que faz parte do Google, chegou, após anos de estudos, a quatro métricas-chave para medir os aspectos de DevOps de um projeto. Suas métricas passam por uma avaliação estatística rigorosa para possibilitar o entendimento da relação entre as medições e o sucesso das organizações (DORA, 2024). 
    \begin{itemize}
        \item Change lead time: Tempo que uma alteração leva para chegar à produção.
        \item Deployment frequency: Frequência com que as alterações chegam à produção.
        \item Change fail percentage: Percentual de implantações que causam falhas em produção.
        \item Failed deployment recovery time: Tempo necessário para se recuperar de uma implantação com falha.
    \end{itemize}  
\end{itemize}



\subsection{How are security practices evaluated?}

É fundamental que as práticas de segurança sejam avaliadas para que seja possível analisar a eficácia das abordagens adotadas e evoluir as já existentes ou adotar novas que se adequem melhor às necessidades da organização. Sendo assim, a avaliação de métricas representa uma forma eficiente e quantitativa de avaliar práticas de segurança, pois permite acompanhar a evolução das atividades e verificar os resultados de cada uma. Outra forma de analisar as práticas de segurança é por meio de auditorias, que verificam a adequação da organização aos padrões de segurança e conformidade \cite{Rajapakse2022}. 

Por fim, podem-se usar modelos de avaliação de maturidade, pois eles introduzem uma linha de base para comparação com a organização avaliada. Eles permitem analisar o estado atual da aplicação, além de identificar áreas de melhoria e traçar um plano para alcançar um nível de segurança mais elevado \cite{Lange2024}.

\subsection{In what ways are security measures evaluated in organizations?}

As medidas de segurança são avaliadas de diversas formas dentro das organizações. Uma das principais é a análise da repercussão das métricas de segurança nos KPIs da organização. KPIs (Indicadores-Chave de Desempenho) são as métricas centrais que medem a saúde dos projetos. É crucial criar alertas e dashboards para acompanhar o desenvolvimento das métricas e KPIs, pois, assim, pode-se obter insights de como as métricas de segurança impactam nos indicadores da empresa, além de possibilitar o rastreio das métricas durante todo o período em que foram monitoradas \cite{Joshi20247}.

Outra maneira de avaliar as medidas de segurança está relacionada ao quanto elas se adequam ao compliance da organização ou a aspectos regulatórios. Muitas vezes, as empresas precisam seguir rígidos padrões de conformidade relacionados às métricas — como a cobertura de testes, que, dependendo da área, precisa ser extremamente alta —, bem como em setores financeiros, onde uma falha de segurança pode gerar um prejuízo bilionário \cite{Kudriavtseva20241223}.

\subsection{What types of vulnerabilities are mitigated by DevSecOps practices?	}

Existem diversos tipos de falhas de segurança que podem ocorrer durante o processo de desenvolvimento de software, entre elas, falhas que podem extrapolar o escopo do trabalho, como as relacionadas ao hardware utilizado. Por esse motivo, é necessário entender quais problemas de segurança são afetados pelas práticas DevSecOps, pois, desse modo, será possível analisar de forma mais assertiva a repercussão da adoção dessa metodologia na qualidade da segurança. Assim, na Tabela 9, foram elencadas as principais vulnerabilidades impactadas por esse paradigma.

\begin{table}[h]
    \centering
    \caption{Vulnerabilidades Reduzidas}
    \begin{tabular}{|p{5cm}|p{8cm}|}
        \hline
        \textbf{Vulnerabilidade} & \textbf{Referência} \\
        \hline
        SQL Injection  & \citeonline{Saeed2025139} \\
        \hline
        Command Injection & \citeonline{Ramirez2020} \\
        \hline
        XSS  & \citeonline{Saeed2025139} \\
        \hline
        XXE & \citeonline{Nocera2023418} \\
        \hline
        Buffer Overflow  & \citeonline{Ramirez2020} \\
        \hline
        CSRF  & \citeonline{Kushwaha2024}  \\
        \hline
        DDoS  & \citeonline{Saeed2025139} \\
        \hline
        MITM  & \citeonline{Nocera2023418} \\
        \hline
        Broken Authentication  & \citeonline{Saeed2025139} \\
        \hline
        Broken Access Control  & \citeonline{Saeed2025139} \\
        \hline
        Security Misconfiguration  & \citeonline{Saeed2025139} \\
        \hline
        Session Hijacking  & \citeonline{Kushwaha2024}  \\
        \hline
        SSRF  & \citeonline{Nocera2023418}  \\
        \hline
    \end{tabular} \\[0.5em]
    \label{tab:vulnerabilidades}
    \small Fonte: Autor
\end{table}



\subsection{What tools and technologies are used in DevSecOps?}
\begin{itemize}
    \item Monitoramento: Prometheus, Grafana, Loki
    \item Infraestrutura: Terraform, Kubernetes, Docker
    \item CI/CD: Jenkins, GitLab CI/CD, GitHub Actions, Tekton, ArgoCD
    \item Testes: SonarQube, FindBugs, Snyk, OWASP Dependency-Check, OWASP ZAP, Trivy, Detect Secrets, Asylo, StackHawk, JMeter, Selenium
\end{itemize}



\subsection{What is the annual volume of DevSecOps publications from 2009 to 2025?}
O volume anual de artigos está representado na Figura 6. Observa-se o crescimento das pesquisas sobre o tema, principalmente após o ano de 2020, chegando a seu pico em 2024. Em abril de 2025, mês em que a string de busca foi executada, a quantidade de estudos já se igualava ao volume de todo o ano de 2023, o que evidencia o crescimento e a importância da área nos meios acadêmico e profissional.
\begin{figure}[h]
    \centering
	\caption{Volume Anual de Artigos entre 2009 e 2025}
    \includegraphics[width=0.9\textwidth]{figuras/volume-anual-artigos.png}

    \label{fig:volume-anual}
	Fonte: Autor
\end{figure}



\subsection{How many articles were published in academic journals?}
Como indicado na Figura 7, a grande maioria dos artigos foi publicada em revistas científicas. Desse modo, sabe-se que a maior parte dos artigos passou por um critério alto de revisão e análise de qualidade, elevando o nível de confiabilidade dos resultados da pesquisa.

\begin{figure}[h]
    \centering
	\caption{Artigos Publicados em Revistas}
    \includegraphics[width=1.0\textwidth]{figuras/artigos-publicados.png}

    \label{fig:artigos-publicados}
	Fonte: Autor
\end{figure}



\subsection{How many studies have experimental validation?}
Um total de dezesseis estudos conta com validação experimental de diferentes tipos, conforme ilustra o gráfico da Figura 8, que apresenta a quantidade de estudos que possuem ou não essa validação. O baixo índice de validação experimental pode estar relacionado à característica emergente da área; por ser muito recente, ainda falta estabelecer e consolidar os métodos de pesquisa comumente usados em outras áreas para a validação dos estudos.

\begin{figure}[h]
    \centering
	\caption{Quantidade de Artigos com Validação Experimental}
    \includegraphics[width=1.0\textwidth]{figuras/quantidade-experimental.png}

    \label{fig:quantidade-experimental}
	Fonte: Autor
\end{figure}



\subsection{If the study has experimental validation, what type?}
Existem diferentes tipos de validação experimental, entre eles o estudo de caso, o experimento, as entrevistas e as pesquisas. Esses foram os métodos utilizados na elaboração dos 16 artigos que contêm validação experimental. A quantidade de cada tipo está representada no gráfico da Figura 9, do qual se depreende que a validação por estudo de caso é, com grande margem, o método mais utilizado. Isso se deve, principalmente, à característica do DevSecOps de estar ligado a muitos projetos reais, tanto na academia quanto na indústria, o que cria um ambiente propício para a aplicação de estudos de caso, pois eles possibilitam analisar e obter insights ao estudar o desenvolvimento realizado durante a construção de um produto.
\begin{figure}[h]
    \centering
	\caption{Tipos de Validação Experimental dos Artigos}
    \includegraphics[width=1.0\textwidth]{figuras/tipos-validacao.png}

    \label{fig:tipos-validacao}
	Fonte: Autor
\end{figure}