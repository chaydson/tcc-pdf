\chapter[Revisão Estruturada da Literatura]{Revisão Estruturada da Literatura}

Este capítulo destina-se a documentar o processo realizado para selecionar o conjunto de obras acadêmicas que compõe a 
bibliografia desta monografia. Para isso, a base de dados Scopus foi escolhida devido à sua característica de indexar 
diversos artigos da área da computação, muitos publicados nos principais meios de divulgação científica (ELSEVIER, 2025).

\section{Protocolo}

O protocolo utilizado para realizar a revisão estruturada da literatura foi baseado no modelo proposto por Kitchenham e Brereton 
(2013). Seu objetivo é tornar possível que outros pesquisadores, partindo do mesmo ponto, cheguem aos mesmos resultados, 
facilitando a replicabilidade em estudos futuros e permitindo a conferência dos resultados obtidos.

\subsection{String de Busca}

A busca em bases de dados acadêmicas requer o uso de um protocolo, pois elas indexam grande quantidade de artigos de várias áreas.
 Seu uso incorreto pode acarretar em um número excessivo de artigos sem relação com o tema da pesquisa ou, inversamente, retornar
  um volume insuficiente de estudos para responder à questão de pesquisa.

Por essa razão, o protocolo PICO foi utilizado para guiar a elaboração de uma string de busca adequada às necessidades da 
monografia. O protocolo, no entanto, precisou ser adaptado, pois sua origem é na medicina e nem todos os seus elementos se 
adequam ao nosso escopo. Uma representação visual que facilita a compreensão do protocolo pode ser vista na 
Figura 4 (PAI et al., 2004).

\begin{figure}[h]
    \centering
	\caption{Abordagem GQM}
    \includegraphics[width=0.7\textwidth]{figuras/pico.png}

    \label{fig:pico}
	Fonte: (PAI et al., 2004)
\end{figure}

Ao adaptar o modelo PICO para o presente trabalho, suas definições adquirem um novo significado no contexto da Engenharia de 
Software. Por exemplo, Patient, outrora usado para indicar o perfil do paciente, passa a representar a área de aplicação, neste 
caso, o desenvolvimento de software. Intervention também sofre adaptação, deixando de significar "tratamento médico" para se 
referir à metodologia ou tecnologia avaliada. Por fim, Outcome (resultado) mantém seu sentido original, referindo-se aos efeitos 
ou consequências observadas.


\section{Seleção dos Artigos}


\section{Resultados}

