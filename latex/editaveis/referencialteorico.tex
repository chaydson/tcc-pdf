\chapter[Referencial Teórico]{Referencial Teórico}

O referêncial teórico é composto pelos estudos e trabalhos que foram utilizados como base para a realização da pesquisa. Desse modo, este capítulo apresenta os conceitos-chave, teorias, modelos e metodologias que guiaram a execução do estudo de caso proposto.

\section{Engenharia de Software Experimental}

A engenharia de software é uma área bastante influênciada pelo comportamento humano, portanto, diferemente de áreas como a física, não é possível a formulação de regras ou leis universais, exceto para certos aspectos estritamente técnicos. No entanto, é necessário realizar escolhas entre diferentes abordagens de desenvolvimento, 
ferramentas, técnicas, práticas e métodos, avaliar ou validar propostas já existentes. Assim, a engenharia de software experimental fornece as ferramentas necessárias para a realização de experimentos e análises, permitindo a validação de propostas e a comparação de diferentes abordagens com base científica. É através da engenharia de software experimental que se tonna possível
compreender e identificar a relação entre diferentes fatores ou variáveis, pois seus resultados são objetivos e estatisticamente significativos. Dessa forma, a engenharia de software experimental oferece uma forma sistemática, controlada e quantificável de avaliar as atividades da área, mesmo que elas sejam baseadas em comportamentos humanos \cite{Wohlin2024}.

O estudo de caso é uma investigação empírica de um fenônemo contemporâneo, ou vários pequenas ocorrências dele, em seu contexto real. Esse tipo de estudo de estudo é adequado para avaliar métodos e ferramentas de engenharia de software, pois permite ao pesquisador observar o fenônemo em tempo real. Devido a esse contexto, seu planejamento, coleta e análise de dados
seguem uma abordagem flexível, o que significa que essas etapas acontecem de forma iterativa e são evoluídas com a maturidade da pesquisa \cite{Wohlin2024}.

Já as \textit{Surveys}, são pesquisas utilizadas para coletar informações sobre pessoas. Com elas é possível descrever, comparar ou explicar suas opiniões e comportamentos, possibilitanto uma compreensão ampla do contexto em análise. Sua execução geralmente acontece por meio de 
questionários ou entrevistas, sendo os questionários a forma principal de coleta de dados. Eles podem coletar dados qualitativos, por meio de perguntas abertas e quantitativos, usando perguntas fechadas, e tem como característica a possibilidade de alcance de um grande número de participantes, fornecendo um panorama geral.
Por sua vez, as entrevistas envolvem o contato direto entre o pesquisador e os participantes, isso permite ao pesquisador fazer perguntas com alto grau de profundidade, o que leva a uma compreensão mais minuciosa do fenônemo. Apesar disso, os custos para realizar uma pesquisa são significamente maiores em relação ao questionário \cite{Wohlin2024}.

Para fundamentar as bases de uma pesquisa baseada na engenharia de software experimental utiliza-se as revisões sistemáticas da literatura. O principal objetivo desse processo é identificar, analisar e interpretar as evidências disponíveis relacionadas a questão
de pesquisa definida pelo pesquisador, de forma livre de viés e através de um protocolo que possibilite a reprodutibilidade. Para alcançar fundamental elaborar um planejamento de a revisão, especificando as questões de pesquisa e o protocolo detalhado de revisão.
Após essa etapa acontece a condução da revisão, etapa em que os estudos são identificados pelas estratégia de busca definida, para que, posteriormente, os estudos seja feita a seleção do material com base nos critérios de inclusão e exclusão. Finalmente, a qualidade 
dos estudos é avaliada para que ao final da revisão os pesquisadores apresentem os achados obtidos com a revisão \cite{Wohlin2024}.

\subsection{Estudo de Caso}

Um estudo de caso é uma metodologia de pesquisa empírica que se concentra na investigação de um fenômeno contemporâneo dentro de seu contexto da vida real, utilizando múltiplas fontes de evidência e bastante adequado para avaliação industrial de métodos e ferramentas de engenharia de software \cite{Wohlin2024}.

Ao conduzir um estudo de caso, existem cinco etapas principais a serem percorridas: definição, preparação para a coleta de dados, coleta de evidências, análise de dados coletados e relato dos resultados \cite{yin2015estudo}.

\subsubsection{Definição}

Define os conceitos fundamentais do estudo de caso que compõe planejamento do estudo a ser realizado.

\begin{itemize}
    \item Objetivo: O que se pretende alcançar. Pode ser gerar descobrir algo sobre algum fenômeno, retratar alguma situação, tentar aprimorar algum aspecto de fenômeno, entre outros. O objetivo tende a ser mais geral no início e evolui durante o estudo.
    \item Caso: O objeto de estudo. Pode ser um projeto de desenvolvimento de software, um grupo de pessoas, um processo, um produto, uma política, um papel na organização, uma tecnologia, entre outros.
    \item Trabalhos Relacionados: São as referências que constituem a base teórica do estudo. Eles são obtidos pela realização de uma revisão da literatura e ajudam o pesquisador conduzir o estudo.
    \item Perguntas de Pesquisa: O que precisa ser conhecido para cumprir o objetivo do estudo. Assim como o objetivo, as perguntas de pesquisa evoluem e são refinadas durante o estudo.
    \item Métodos: As decisões principais sobre os métodos de coleta de dados são definidas nesta fase.
    \item Seleção: Escolhe o caso e as unidades de análise que serão analisados. 
  \end{itemize}


\subsubsection{Preparação e Coleta de Dados}

  Estabelece quais dados serão coletados e de que forma isso acontecerá. É ncessário estabelecer a triangulação dos dados para aumentar confiabilidade dos resultados obtidos.

  \begin{itemize}
    \item Primeiro grau: O pesquisador tem contato direto com os participantes. 
    \item Segundo grau: O pesquisador coleta dados brutos diretamente sem interagir com os sujeitos durante a coleta, como o monitoramento automático de ferramentas de engenharia de software.
    \item Terceiro grau: O pesquisador utiliza dados já disponíveis e tratados.
  \end{itemize}


\subsubsection{Análise de Dados Coletados}

Os dados coletados, sejam qualitativos ou quantitativos, são analisados visando obter informações relevantes para o estudo de caso.  

\begin{itemize}
    \item Análise de Dados Quantitativos: Inclui estatísticas descritivas, análise de correlação, desenvolvimento de modelos preditivos e teste de hipóteses.
    \item Análise de Dados Qualitativos: Visa extrair conclusões a partir dos dados.
  \end{itemize}


\subsubsection{Relato dos Resultados}

Aqui são apresentados os resultados que os pesquisadores conseguiram obter com a aplicação do estudo de caso. 

\begin{itemize}
    \item Contexto: apresenta a visão geral do projeto e informações pertinentes relacionadas ao ambiente onde o estudo de caso foi realizado.
    \item Procedimentos: aborda as etapas realizadas ao decorrer do estudo.
    \item Resultados: apresenta os resultados de forma clara e com evidências sólidas.
    \item Discussão: os resultados são interpretados e são apresentadas asas limitações do estudo e suas implicações.
    \item Conclusões: Sintetíza os achados da pesquisa.
  \end{itemize}
  
\section{Modelos de Qualidade de Software}
 
ISO 25010

McCall e Boehm

Dromey e Coleman

SIG e SQALE

Quamoco e Q-Rapids

QATCH

\section{Segurança de Software}

A segurança de software é um aspecto fundamental da qualidade de software, sendo um dos atributos principais do modelo de qualidade \citeonline{ISOIEC25010}. 
Ela é definida como o grau em que um sistema ou produto de software protege informações e dados, armazenados ou em trânsito, de forma a impedir o acesso não autorizado por diferentes entes, sejam pessoas, programas ou outros sistemas, de acordo
com os seus respectivos níveis de permissão, assim contribuindo para a confiança do sistema \cite{ISOIEC25010}. 

Para possibilitar a sua análise detalhada, a \citeonline{ISOIEC25010} define cinco sub-características de segurança fundamentais. A primeira delas é a confidencialidade, que se refere à capacidade do sistema de proteger informações contra acessos não autorizados. 
A segunda é a integridade, que diz respeito à capacidade do sistema de proteger informações contra modificações não autorizadas. A terceira sub-característica é o não-repúdio, que se refere à capacidade do sistema de garantir que as ações realizadas por um usuário ou sistema não possam ser negadas posteriormente. 
A quarta sub-característica é a responsabilidade, que diz respeito à capacidade do sistema de rastrear e registrar as ações realizadas por usuários ou sistemas. Por fim, a quinta sub-característica é a autenticidade, que se refere à capacidade do sistema de garantir que os usuários ou sistemas são quem dizem ser.


\section{DevOps}

A origem do termo e do movimento DevOps remonta a dois eventos distintos no ano de 2009. O primeiro deles aconteceu em julho daquele ano, quando John Allspaw e Paul Hammond apresentaram a palestra "10 deploys por dia: cooperação entre Dev \& Ops no Flickr" na conferência Velocity. 
Poucos meses depois, Patrick Debois organizou a conferência "DevOps Days" em Ghent, Bélgica, que é considerada o primeiro evento oficial do movimento DevOps. Desde então, o termo e o movimento DevOps ganharam popularidade e se tornaram uma abordagem amplamente adotada para o desenvolvimento e operação de software \cite{DORA2024}.

Em suma, DevOps é um conjunto de práticas e valores culturais que visa encurtar o ciclo de vida de desenvolvimento de software, aumentar a qualidade do software produzido e facilitar a sua evolução contínua. Para alcançar esses objetivos, o DevOps promove a colaboração e a comunicação entre as equipes de desenvolvimento e operações \cite{Aljohani2023}.

\citeonline{Sinan2025} definem como práticas basilares a automação e o monitoramento. A automação é essencial para realizar tarefas repetitivas de forma eficiente e consistente, reduzindo erros humanos e aumentando a velocidade de entrega. 
Já o monitoramento é fundamental para garantir o funcionamento correto, desempenho e segurança dos sistemas em produção, permitindo a identificação e resolução rápida de problemas. Ademais, essas duas práticas se beneficiam mutuamente, uma vez que o monitoramento possibilidade observar o comportamento da automação e a automação permite que o processo de monitoramento seja mais eficiente.

Segundo \citeonline{Sinan2025}, o pipeline de CI/CD é a implementação prática dos princípios de automação e monitoramento. Ele funciona como uma esteira de produção de software, onde o código é continuamente integrado, testado e entregue de forma automatizada. 
O pipeline de CI/CD é composto por duas etapas princípais, a Integração Contínua (CI) e a Entrega Contínua (CD). 

A primeira fase é a Integração Contínua, aqui o código é frequentemente integrado em um repositório central compartilhado, onde é automaticamente testado para garantir que ele funcione corretamente. 
A etapa seguinte é a Entrega Contínua. Nessa fase, o código que passou pelos testes na etapa de CI é automaticamente implantado no ambientes de produção e entregue aos usuários finais \cite{Sinan2025}.

\section{DevSecOps}

Nos modelos tradicionais de desenvolvimento de software, a segurança é frequentemente tratada como uma etapa separada, realizada por uma equipe isolada e realizada no fim do ciclo de vida do desenvolvimento. Isso pode levar a atrasos na entrega do software, aumento dos custos e vulnerabilidades de segurança não detectadas.
Essa situação é ainda mais acentuada em ambientes DevOps, onde a velocidade e a integração contínua são priorizadas e essa metodologia acaba representando um gargalo no processo de desenvolvimento \cite{Sinan2025}.

Para superar esses desafios, surgiu o conceito de DevSecOps, que integra práticas de segurança diretamente no ciclo de vida do desenvolvimento de software. O objetivo do DevSecOps é garantir que a segurança seja uma responsabilidade compartilhada entre as equipes de desenvolvimento, operações e segurança, promovendo uma cultura de colaboração e comunicação contínua \cite{Sinan2025}.

Assim, a implementação de práticas DevSecOps consiste na inserção de ferramentas e processos de segurança em diferentes etapas do pipeline de CI/CD. Essa prática ficou conhecida como deslocar a segurança para a esquerda (\textit{shift-left security}), que significa incorporar a segurança já nas fases iniciais do desenvolvimento.

Como primeira barreira de defesa, a análise estática de código (\textit{Static Application Security Testing} - SAST) é utilizada para identificar vulnerabilidades diretamente no código-fonte, abordagem conhecida como caixa-branca. Ferramentas SAST analisam o código em busca de padrões inseguros, práticas de codificação inadequadas e possíveis falhas de segurança.

Além disso, a análise dinâmica de código (\textit{Dynamic Application Security Testing} - DAST) é empregada para testar a aplicação em execução, simulando ataques reais e identificando vulnerabilidades que podem ser exploradas por agentes maliciosos. Essa abordagem é conhecida como caixa-preta, pois ela testa a aplicação em execução \cite{Sinan2025}.

O modelo de maturidade de segurança OWASP DSOMM (\textit{DevSecOps Maturity Model}) é uma importante 
ferramenta na construção e avaliação de projetos \textit{DevSecOps}. Ele define atividades, métricas e tecnologias 
que devem ser usadas para construir um ambiente \textit{DevSecOps}, além de proporcionar o acompanhamento da 
maturidade da segurança do projeto em cinco dimensões: \textit{Build and Deployment}, \textit{Culture and Organization}, 
\textit{Implementation}, \textit{Information Gathering} e \textit{Test and Verification} \cite{Lange2024}.
