\chapter[Referencial Teórico]{Referencial Teórico}

\section{Engenharia de Software Experimental}

Introdução

Resumo Revisão Sistemática da Literatura

Resumo Estudo de Caso

Resumo Questionário

\subsection{Estudo de Caso}


\section{Modelos de Qualidade de Software}

\section{Segurança de Software}

\section{DevOps}

\section{DevSecOps}

Nesse sentido, o modelo de maturidade de segurança OWASP DSOMM (\textit{DevSecOps Maturity Model}) é uma importante 
ferramenta na construção e avaliação de projetos \textit{DevSecOps}. Ele define atividades, métricas e tecnologias 
que devem ser usadas para construir um ambiente \textit{DevSecOps}, além de proporcionar o acompanhamento da 
maturidade da segurança do projeto em cinco dimensões: \textit{Build and Deployment}, \textit{Culture and Organization}, 
\textit{Implementation}, \textit{Information Gathering} e \textit{Test and Verification} \cite{Lange2024}.
