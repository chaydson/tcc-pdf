\chapter[Introdução]{Introdução}

Neste capítulo são apresentados os conceitos fundamentais que norteiam este trabalho: segurança de produtos de software; Modelos de Qualidade; DevOps e; for fim; DevSecOps. 
Adicionalmente, são apresentados o escopo do problema, a questão de pesquisa que guia essa investigação 
e a estrutura geral do documento.

\section{Contexto}

Avaliar os fatores de qualidade de um \textit{software} é de suma importância no desenvolvimento de \textit{software} e, 
para isso, faz-se necessária a utilização de um modelo que guie a avaliação, de forma a sistematizar o 
processo e reduzir subjetividades \cite{Siavvas21}. Nesse sentido, o modelo proposto por \citeonline{McCall77}, foi o primeiro modelo proposto para analisar a qualidade de produto de software. Nesse trabalho, foram identicados aspectos e  propriedades do produto de software, chamados de fatores e seus respectivos subfatores. Também foi proposto um método de avaliação quantitativa, baseada nos fatores, critérios e métricas. Nesse modelo, aspetos relacionados à segurança, foi percebido como um subfator de Integridade. Tratou-se portanto, de um modelo hierárquico. Subsequentemente, 
o modelo proposto por \citeonline{boehm1978characteristics} evoluiu as ideias de McCall, e ambos se tornaram modelos seminais, servindo de referência para os modelos subsequentes. Além de propor novas propriedades e aspectos, renomeou os fatores e subfatores para características e subcaracterísticas. Essa terminologia é utitilzada até hoje, em modelos de referência de qualidade contemporâneos.

A partir da \citeonline{ISOIEC9126}, estabeleceu-se um uma norma padrão internacional para analisar a qualidade de produto de software. Essa norma foi baseada nos modelos de \citeonline{McCall77} e \citeonline{boehm1978characteristics} além de outros, como por exemplo o modelo de \citeonline{Dromey:1995:MSP:205302.205311}, que incoporava aspectos e propriedades específicas de código-fonte baseado no paradigma orientado a objetos-OO. Na \citeonline{ISOIEC9126} a segurança foi definida como uma subcaracterística de Funcionalidade. A \citeonline{ISOIEC25010} surgiu como uma evolução da ISO 9126, expandindo seus conceitos e adaptando o modelo para a realidade da qualidade de produto de software modernos. Uma relevante alteração foi transformação da até então, subcaracteristica de qualidade para característica, destacando-a como um dos pilares da qualidade do produto. Além de, reorganizar as características e suas respectivas subcaracteríticas na visão de qualidade interna, também foi incorporada ao modelo a visão da qualidade em uso.
Tem-se ainda, a \citeonline{isoiec27001}, que define requisitos para implementar, manter e melhorar continuamente, a segurança da informação em uma organização; além, da \citeonline{isoiec27034}, que fornece diretrizes para as organizações sobre como integrar a segurança em suas aplicações, ao longo de todo o ciclo de vida de desenvolvimento, busacando  mitigar as potenciais ameaças cibernéticas. Contudo, nenhuma dessas normas ISO/IEC, fornecem um método para avaliar a 
segurança de software de maneira quantitativa ou qualitativa. Assim, torna-se necessário recorrer a outros modelos e 
ferramentas que ofereçam uma abordagem para operacionalizar os conceitos abstratos definidos  nessas normas, com  suas respectivas fórmulas de cálculo, medidas e métricas, valores de referência, ponderação e agregação numérica, possibilitando o cálculo de indicadores quantitativos e qualitativos de segurança. Entretanto, é importante destacar que existem alguns frameworks  e práticas que auxiliam a observação de aspcetos técnicos da segurança, quanto organizaçionais, como por exemplo:
\begin{itemize}
    \item o modelo de maturidade \textit{OWASP-DSOMM} \cite{owaspsamm} e a lista das dez vulnerabilidades de maior ocorrência \textit{OWASP Top 10} \cite{owasptop10_2021};
     \item o catálogo de vulnerabilidades \textit{Common Weakness Enumeration-CWE}, mantido por uma comunidade que envolve a indústria, academia e governo norte americano \cite{cwe}
\end{itemize}

Nos dias atuais, a segurança de sistemas e produtos de software é um dos fatores mais críticos. Os ataques cibernéticos explorando vulnerabilidades de produtos de software tem provocado impactos negativos, em ordem mundial. Recente, no Brasil, por exemplo, vivenciamos  a tida como maior invasão de dispositivo eletrônico do País, provocando prejuízos financeiros na ordem  R\$ 800 milhões \cite{uolpix} \cite{g1pix} \cite{vepix}. Portanto, torna-se imperativo promover a  cultura de desenvolvimento seguro, além de, incorporar práticas e técnicas de segurança, que tornem as análises e decisões mais sistemáticas, ao longo de todo ciclo de desenvolvimento.

Nas últimas três décadas, a Engenharia de Software passou por uma profunda transformação na mameira de como se desenvolver produtos de software. Isso se deu por meio dos métodos ágeis, a filosofia de desenvolvimento Lean e práticas de comunidades de software livre \cite{Fitzgerald2017} \cite{LOPEZ2022111187}. Atualmente, a cadência do ciclo de desenvolvimento acontece em um pequeno intervalo de tempo (dias, semanas). Isso fez com que entrega e implantação de versões de produtos de software passassem a ser disponibilizas e implantadas, continuamente. Além disso, os antigos sistemas "monolito", transformaram em subsistemas com responsabilidades específicas, especializadas e, que se comunicam entre si. Assim, as versões dos produtos passaram a ser cada vez "menores" e independentes do ponto devista da implantação. Para lidar com essa nova realidade escala no desenvolvimento de produtos de software, é necessário dispormos de tecnologias que nos auxiliem com a automação e sistematização das análises, com efeito na tomada de decisão sobre novas implantações.

Na esteira da transformação ágil, em 2009, Patrick Debois propôs o termo DevOps. Esse termo abarcou um conjunto de nuances que que em essência, buscava remover a barreira existente entre os times de desenvolvimento e operações, quando da implantação. A compreensão sobre esse termo passa por diferentes facetas como: cultura, automação; métodos e procedimentos de desenvolvimento e operações contínuos e integrados; colaboração \cite{10.1145} \cite{10.21203v1} \cite{DBLP:journals/jss/LuzPB19} . Um dos principais objetivos é reduzir o ciclo 
de vida do desenvolvimento de software, permitindo entregas mais frequentes e automatizadas ou semi-automatizadas. As principais práticas de 
DevOps são a Integração Contínua (CI), que consiste em integrar o código desenvolvido na ramificação 
principal com validação de build e testes de forma automática para detectar falhas, e a Entrega/Implantação
Contínua (CD), que consiste em deixar o software pronto para entrar em produção e realizar o seu 
lançamento de forma automatizada \cite{Rajapakse2022}. 

Já o \textit{DevSecOps} integra os princípios e práticas do \textit{DevOps}, adicionando o time de segurança ao processo. 
Esse paradigma implementa uma abordagem de segurança chamada \textit{Shift-Left}, na qual os processos de segurança
são realizados desde o início do desenvolvimento, com o objetivo de evitar problemas decorrentes de uma 
avaliação tardia. Além disso, é formado por práticas de segurança como treinamento da equipe, 
testes de segurança automatizados e feedback contínuo \cite{Rajapakse2022}.

\section{Problema}

Medir a segurança de software representa um grande desafio \cite{Rajapakse2022}. Embora já exista muito conhecimento acumulado na área, ainda carecemos de modelos que apresentem 
formas sistematizadas de avaliação da segurança. Frequentemente, os métodos são baseados em critérios subjetivos, como a opnião de especialistas e não possuem validação empírica, o que afeta a 
confiabilidade dos resultados \cite{Siavvas21}.

No contexto do desenvolvimento e implantação contínua, esse desafio se torna ainda mais difícil. Métodos tradicionais de análise de segurança 
são impraticáveis devido à velocidade das entregas \cite{Rajapakse2022}. A medição da seguranca se torna ainda mais desafiadora ao lidar ciclos continuos de lançamentos de versões de produtos. A segurança é uma propriedade multifacetada, 
emergente e dependente do contexto, o que complica sua quantificação \cite{Kudriavtseva20241223}.
Em essência, a falta de compreensão sobre cultura e práticas de desenvolvimento contínuo e seguro, pode comprometer ou mesmo inviabilizar, a estratégia de negócio de várias organizações mundiais.

\section{Questão de Pesquisa}
\label{qp_principal}

A definição da questão de pesquisa foi elaborada utilizando a abordagem \textit{Goal 
Question Metric} (GQM). Essa é uma abordagem que tem como objetivo definir, de 
maneira \textit{top-down} e hierárquica, os objetivos a serem alcançados, as perguntas a 
serem respondidas para cumprir tais objetivos e as métricas necessárias para 
responder a cada pergunta de forma quantitativa, como mostra a Figura \ref{fig:gqm}. Essa 
estrutura foi adaptada para o contexto da pesquisa, conforme a Tabela \ref{tab:gqm-adaptado}, 
resultando na seguinte questão:

\begin{table}[h]
    \centering
	\caption{GQM Adaptado}
    \begin{tabular}{|p{5cm}|p{9cm}|}
        \hline
        \textbf{Característica} & \textbf{Valor} \\
        \hline
        \textbf{Analisar} & a característica de qualidade de produto de software com foco na segurança e suas respectivas subcaracterísticas, nas visões interna e externa. \\ 
        \hline
        \textbf{Com o propósito de} & Caracterizar \\
        \hline
        \textbf{Com respeito ao} & desenvolvimento contínuo de produtos de software seguros (DevSecOps) \\
        \hline
        \textbf{Do ponto de vista de} & Pesquisador \\
        \hline
        \textbf{No contexto do} & desesenvolvimento de aplicações web seguras e de código-fonte aberto\\
        \hline
    \end{tabular} \\[0.5em]
   	Fonte: Adaptado de \citeonline{Basili1994}
	\label{tab:gqm-adaptado}
\end{table}

Considerando essa perspectiva definimos a sguinte questão principal de pesquisa deste estudo

\begin{center} 
\textbf{\textit{Como analisar a característica de segurança no desenvolvimento contínuo de sistemas web, considerando as 
visões de qualidade interna e externa}}
\end{center}

\begin{figure}[h]
    \centering
	\caption{Abordagem GQM}
    \includegraphics[width=0.9\textwidth]{figuras/GQM.png}

    \label{fig:gqm}
	Fonte: Adaptado de \citeonline{Basili1994}
\end{figure}

\section{Objetivos}

O objetivo geral deste trabalho consiste em analisar a adoção de práticas de DevSecOps no ciclo de devida de desenvolvimento do produto de software-livre \textcolor{blue}{MEPA}\footnote{\href{https://gitlab.com/lappis-unb/projetos-energia/mec-energia}{Link para o repositório do projeto}}. Para alcançar este propósito, 
foram definidos os seguintes objetivos específicos:

\begin{itemize}
    \item Fundamentar teoricamente os conceitos de \textit{DevSecOps}, modelos de segurança e metodologias de desenvolvimento seguro.
    \item Incorporar um conjunto de práticas \textit{DevSecOps} ao ciclo de desenvolvimento do produto de \textit{software} sob análise.
    \item Planejar um estudo de caso focado na observação das práticas implementadas.
    \item Conduzir o estudo de caso, realizando a coleta e a análise das medidas e métricas segurança, que apoiem a discussão dos resultados alcançados.
    \item Apresentar as conclusões e os \textit{insights} resultantes desta investigação.
\end{itemize}

\section{Estrutura do Trabalho}

A seguir, são apresentados os capítulos que compõem a estrutura deste trabalho.

\begin{itemize}
    \item Introdução: apresenta a contextualização do trabalho, o problema de pesquisa, a definição da questão de pesquisa e dos objetivos do trabalho. Por fim, descreve a estrutura das atividades realizadas.
     \item Referencial Teórico: estabelece a fundamentação teórica da monografia, abordando os tópicos centrais da pesquisa: DevSecOps, qualidade de software, segurança de software e modelos de avaliação de maturidade.
    \item Revisão Estruturada da Literatura: explicita o processo empregado para seleção que fundamentam este trabalho incluindo o protocolo de pesquisa e filtragem dos estudos e os resultados obtidos.
    \item Estudo de caso: descreve o protocolo utilizado para a condução do estudo de caso, detalhando seus objetivos, perguntas de pesquisa, atividades e resultados alcançados.
    \item Conclusão: consolida os achados obtidos parcial ou integralmente e como esses resultados respondem à questão de pesquisa principal, bem como as limitações do estudo e as possibilidades de aprofundamento de trabalhos futuros.
\end{itemize}

\section{Cronograma e Atividades}
\label{sec:cronograma-atividaes}

\subsection{Primeira Etapa}
Esta subseção detalha as atividades desenvolvidas na primeira etapa da monografia. A Figura \ref{fig:fluxograma} ilustra o 
fluxo das atividades, enquanto a Figura \ref{fig:cronograma} apresenta o cronograma correspondente.

\begin{figure}[h]
    \centering
	\caption{Atividades da Primeira Etapa}
    \includegraphics[width=1.0\textwidth]{figuras/fluxo-atividades.png}

    \label{fig:fluxograma}
	Fonte: Autor
\end{figure}

\begin{figure}[h]
    \centering
	\caption{Cronograma da Primeira Etapa}
    \includegraphics[width=1.0\textwidth]{figuras/cronograma.png}

    \label{fig:cronograma}
	Fonte: Autor
\end{figure}

\begin{itemize}
    \item Contextualização sobre Engenharia de Software Experimental: Estudo sobre os métodos de pesquisa empírica em Engenharia de Software, como revisão sistemática da literatura, survey, experimentos e estudo de caso. Conceitos esses, fundamentais para a condução do trabalho proposto. \cite{Wohlin:2012:ESE:2349018} \cite{yin2015estudo} \cite{10.1007/s10664-008-9102-8}
    \item Definição do GQM: Aplicação da abordagem Goal Question Metric (GQM) \cite{gqm} para a construção da questão principal de pesquisa, suas subquestões e as medidas a serem analisadas.
    \item Elaboração do Protocolo de Revisão da Literatura: Estruturação e adaptação do protocolo de revisão sistemática da literatura proposto por \citeonline{kitchenham2007guidelines}. Essa sistematização organizará nosso processo de revivão da literatura pertinente. Este processo inclui a definição da string de busca baseada no protocolo PICO, adapatado da área da medicina \cite{pai_clinical_2004}. Por meio desse protocolo definimos sinônimos de termos, critérios de inclusão e exclusão de estudos, além do método de análise  e  extração de dados.
    \item Seleção dos Artigos: Execução da filtragem dos estudos por meio da leitura de títulos, resumos e palavras-chave, com a aplicação dos critérios de inclusão e exclusão definidos.
    \item Análise do Material Selecionado: Leitura completa dos artigos selecionados para aprofundar o conhecimento sobre o estado da arte em métodos de avaliação de segurança de sistemas web e práticas de desenvolvimento seguro.
    \item Definição da Proposta de Solução: Definição os modelos de segurança, as ferramentas e as atividades do estudo de caso.
    \item Redação da Monografia: Escrita do texto da monografia, conforme a estrutura presente na Seção 1.5.
    \item Revisão: Realização das correções e dos ajustes solicitados pelo orientador.
    \item Defesa: Preparação do material e apresentação do trabalho final para a banca examinadora.
\end{itemize}

