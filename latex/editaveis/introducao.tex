\chapter[Introdução]{Introdução}

Este capítulo introduz os conceitos fundamentais que norteiam este trabalho: 
experimentação em Engenharia de Software, segurança de produtos de software e DevSecOps. 
Adicionalmente, são apresentados o escopo do problema, a questão de pesquisa que guia a investigação 
e a estrutura geral do documento e das atividades.

\section{Contexto}

Avaliar os fatores de qualidade de um software é de suma importância no desenvolvimento de software e, 
para isso, faz-se necessária a utilização de um modelo que guie a avaliação, de forma a sistematizar o 
processo e reduzir subjetividades (SIAVVAS et al., 2021). Nesse sentido, o modelo proposto por 
McCall (MCCALL; RICHARDS; WALTERS, 1977) foi o primeiro modelo hierárquico para analisar a qualidade, 
no qual os diferentes fatores eram analisados por critérios e avaliados por métricas. Subsequentemente, 
o modelo de Boehm (BOEHM, 1978) evoluiu as ideias de McCall, e ambos se tornaram modelos seminais, 
servindo de referência para os modelos subsequentes.

A partir da ISO/IEC 9126 (2001), estabeleceu-se um padrão internacional para a qualidade de software que 
decompunha essa qualidade em características e subcaracterísticas, em um modelo hierárquico, além de 
definir os termos técnicos da área. A ISO/IEC 25010 (2011) surgiu como uma evolução da ISO 9126, expandindo
seus conceitos e adaptando o modelo para a nova realidade da qualidade de software moderno. Diferentemente
da ISO 9126, na ISO 25010 a segurança já é definida como um dos pilares da qualidade,
e não como uma subcaracterística.

Como padrão internacional de segurança da informação, tem-se a ISO/IEC 27001 (2022). 
Ela se difere das normas de qualidade de software na medida em que seu foco está na especificação dos 
requisitos necessários para estabelecer, implementar, manter e melhorar continuamente um Sistema de Gestão 
de Segurança da Informação (SGSI).

Contudo, tanto a ISO/IEC 25010 (2011) quanto a ISO/IEC 27001 (2022) não fornecem um método para avaliar a 
segurança de software de maneira quantitativa. Assim, torna-se necessário recorrer a outros modelos e 
ferramentas que ofereçam uma abordagem numérica para medir a segurança, por meio da definição de métricas, 
do seu cálculo e do estabelecimento de valores de referência para avaliá-las.


O DevOps é um paradigma que visa remover as barreiras entre os times de desenvolvimento e operações, 
a fim de construir um ambiente colaborativo e integrado (RAJAPAKSE et al.). Seu objetivo é reduzir o ciclo 
de vida do desenvolvimento de software, permitindo entregas mais frequentes. As principais práticas de 
DevOps são a Integração Contínua (CI), que consiste em integrar o código desenvolvido na ramificação 
principal com validação de build e testes de forma automática para detectar falhas, e a Entrega/Implantação
Contínua (CD), que consiste em deixar o software pronto para entrar em produção e realizar o seu 
lançamento de forma automatizada (RAJAPAKSE et al.).

Já o DevSecOps integra os princípios e práticas do DevOps, adicionando o time de segurança ao processo. 
Esse paradigma implementa uma abordagem de segurança chamada Shift-Left, na qual os processos de segurança
são realizados desde o início do desenvolvimento, com o objetivo de evitar problemas decorrentes de uma 
avaliação tardia. Além disso é formado por práticas de segurança como treinamento da equipe, 
testes de segurança automatizados e feedback contínuo (RAJAPAKSE et al.).

Nesse sentido, o modelo de maturidade de segurança OWASP DSOMM (DevSecOps Maturity Model) é uma importante 
ferramenta na construção e avaliação de projetos DevSecOps. Ele define atividades, métricas e tecnologias 
que devem ser usadas para construir um ambiente DevSecOps, além de proporcionar o acompanhamento da 
maturidade da segurança do projeto em cinco dimensões: Build and Deployment, Culture and Organization, 
Implementation, Information Gathering e Test and Verification (LANGE; KUNZ, 2024).

\section{Problema}

Medir a segurança de software representa um grande desafio(RAJAPAKSE et al.). A literatura atual carece de modelos que apresentem 
formas sistematizadas de avaliação da segurança; frequentemente, os métodos são baseados em critérios 
subjetivos, como a análise manual por especialistas, e não possuem validação empírica, o que afeta a 
confiabilidade dos resultados (SIAVVAS et al., 2021).

No contexto DevOps, esse desafio se torna ainda mais difícil. Métodos tradicionais de análise de segurança 
são impraticáveis devido à velocidade das entregas (RAJAPAKSE et al.). A medicao da seguranca se torna ainda
mais desafiador ao lidar ciclos continuos de lançamento. A segurança é uma propriedade multifacetada, 
emergente e dependente do contexto, o que complica sua quantificação (KUDRIAVTSEVA; GADYATSKAYA, 2024).

\section{Questão de Pesquisa}

A definição da questão de pesquisa foi elaborada utilizando a abordagem Goal 
Question Metric (GQM). Essa é uma abordagem que tem como objetivo definir, de 
maneira top-down e hierárquica, os objetivos a serem alcançados, as perguntas a 
serem respondidas para cumprir tais objetivos e as métricas necessárias para 
responder a cada pergunta de forma quantitativa, como mostra a Figura 1. Essa 
estrutura foi adaptada para o contexto da pesquisa, conforme a Tabela 1, 
resultando na seguinte questão:

\begin{table}[h]
    \centering
	\caption{GQM Adaptado}
    \begin{tabular}{|p{5cm}|p{9cm}|}
        \hline
        \textbf{Característica} & \textbf{Valor} \\
        \hline
        Analisar & A característica de qualidade de produto de software: \textbf{segurança} \\
        \hline
        Subcaracterísticas & Confidencialidade, integridade, autenticidade, responsabilidade, etc. \\
        \hline
        Visões & Interna e externa \\
        \hline
        Com o propósito de & Caracterizar \\
        \hline
        Com respeito a & Desenvolvimento e operação de produtos de software seguros (DevSecOps) \\
        \hline
        Do ponto de vista de & Pesquisador \\
        \hline
        No contexto de & Desenvolvimento de aplicações web seguras (software livre, organizações públicas e privadas) \\
        \hline
    \end{tabular} \\[0.5em]
   	Fonte: Adaptado de BASILI, CALDIERA e ROMBACH (1994)
	\label{tab:gqm-adaptado}
\end{table}

\begin{center}
Como analisar a característica de segurança no desenvolvimento contínuo de sistemas web, considerando as 
visões de qualidade interna e externa? 
\end{center}

\begin{figure}[h]
    \centering
	\caption{Abordagem GQM}
    \includegraphics[width=0.7\textwidth]{figuras/GQM.png}

    \label{fig:gqm}
	Fonte: Adaptado de BASILI, CALDIERA e ROMBACH (1994)
\end{figure}

\section{Objetivos}

O objetivo geral deste trabalho consiste em avaliar o impacto das práticas DevSecOps na qualidade interna 
e externa de um produto de software, por meio de uma análise quantitativa. Para alcançar este propósito, 
foram definidos os seguintes objetivos específicos:

\begin{itemize}
    \item Fundamentar teoricamente os conceitos de DevSecOps, modelos de segurança e metodologias de desenvolvimento seguro.
    \item Incorporar um conjunto de práticas DevSecOps ao ciclo de desenvolvimento do produto de software sob análise.
    \item Planejar um estudo de caso focado na observação das práticas implementadas.
    \item Conduzir o estudo de caso, realizando a coleta e a análise das métricas de qualidade de software.
    \item Apresentar as conclusões e os insights resultantes desta investigação.
\end{itemize}

\section{Estrutura do Trabalho}

A seguir, são apresentados os capítulos que compõem a estrutura deste trabalho.

\begin{itemize}
    \item Introdução: apresenta a contextualização do trabalho, o problema de pesquisa, a definição da questão de pesquisa e dos objetivos do trabalho. Por fim, descreve a estrutura das atividades realizadas.
    \item Revisão Estruturada da Literatura: explicita o processo empregado para seleção que fundamentam este trabalho incluindo o protocolo de pesquisa e filtragem dos estudos e os resultados obtidos.
    \item Referencial Teórico: estabelece a fundamentação teórica da monografia, abordando os tópicos centrais da pesquisa: DevSecOps, qualidade de software, segurança de software e modelos de avaliação de maturidade.
    \item Estudo de caso: descreve o protocolo utilizado para a condução do estudo de caso, detalhando seus objetivos, perguntas de pesquisa, atividades e resultados alcançados.
    \item Conclusão: consolida os achados obtidos ao final do estudo e como esses resultados respondem à questão de pesquisa principal, bem como as limitações do estudo e as possibilidades de aprofundamento de trabalhos futuros.
\end{itemize}

\section{Cronograma e Atividades}

\subsection{Primeira Etapa}
Esta subseção detalha as atividades desenvolvidas na primeira etapa da monografia. A Figura 2 ilustra o 
fluxo das atividades, enquanto a Figura 3 apresenta o cronograma correspondente.

\begin{figure}[h]
    \centering
	\caption{Atividades da Primeira Etapa}
    \includegraphics[width=1.0\textwidth]{figuras/fluxo-atividades.png}

    \label{fig:fluxograma}
	Fonte: Autor
\end{figure}

\begin{figure}[h]
    \centering
	\caption{Cronograma da Primeira Etapa}
    \includegraphics[width=1.0\textwidth]{figuras/cronograma.png}

    \label{fig:cronograma}
	Fonte: Autor
\end{figure}

\begin{itemize}
    \item Contextualização sobre Engenharia de Software Experimental: Estudo sobre os métodos de pesquisa empírica em Engenharia de Software, como revisão sistemática da literatura, survey, experimentos e estudo de caso (WOHLIN et al., 2024), fundamentais para a condução do trabalho.
    \item Definição do GQM: Aplicação da abordagem Goal Question Metric (GQM) (BASILI; CALDIERA; ROMBACH, 1994) para a construção da questão principal de pesquisa, suas subquestões e as métricas que orientarão a revisão da literatura.
    \item Elaboração do Protocolo de Revisão: Estruturação de um protocolo de revisão sistemática da literatura (KITCHENHAM; BRERETON, 2013) para pesquisar, selecionar e analisar os artigos. Este processo inclui a definição da string de busca (baseada no framework PICO), a elaboração de sinônimos, a definição dos critérios de inclusão e exclusão e o método para extração de dados.
    \item Seleção dos Artigos: Execução da filtragem dos estudos por meio da leitura de títulos, resumos e palavras-chave, com a aplicação dos critérios de inclusão e exclusão definidos.
    \item Análise do Material Selecionado: Leitura completa dos artigos selecionados para aprofundar o conhecimento sobre o estado da arte em métodos de avaliação de segurança de sistemas web e práticas de desenvolvimento seguro.
    \item Definição da Proposta de Solução: Definição os modelos de segurança, as ferramentas e as atividades do estudo de caso.
    \item Redação da Monografia: Escrita do texto da monografia, conforme a estrutura presente na Seção 1.5.
    \item Revisão: Realização das correções e dos ajustes solicitados pelo orientador.
    \item Defesa: Preparação do material e apresentação do trabalho final para a banca examinadora.
\end{itemize}
