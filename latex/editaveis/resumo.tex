\begin{resumo}
    O presente trabalho tem como objetivo analisar a adoção de práticas de DevSecOps no ciclo de vida de desenvolvimento do produto de software livre Brasil Participativo, investigando os impactos na segurança do projeto. 
    Desse modo, foi realizado um estudo de caso, estruturado pela abordagem GQM e fundamentado por uma revisão da literatura. 
    O método envolveu a integração de ferramentas de análise estática, dinâmica e de composição de software na pipeline de integração contínua para a coleta automatizada de dados quantitativos. 
    Os resultados indicaram que a automação proporcionou a identificação eficaz de vulnerabilidades nas perspectivas de qualidade interna e externa, apesar de a pipeline ter apresentado indicadores iniciais de baixa performance devido à alta taxa de falhas detectadas. 
    Conclui-se que a incorporação de práticas de DevSecOps viabilizou a detecção precoce de falhas e forneceu parâmetros quantitativos fundamentais para a tomada de decisão, reforçando a importância do monitoramento contínuo em ambientes de desenvolvimento ágil.
 \vspace{\onelineskip}
    
 \noindent
 \textbf{Palavras-chave}: DevSecOps. Segurança de Software. Estudo de Caso. CI/CD. Qualidade de Software
\end{resumo}
