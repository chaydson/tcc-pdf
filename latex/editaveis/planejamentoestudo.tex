\chapter[Planejamento do Estudo de Caso]{Planejamento do Estudo de Caso}

Neste capítulo é apresentado o planejamento do estudo de caso conduzido na segunda etapa do trabalho, aqui são estabelecidos os conceitos e definições relacionados a esse tipo de estudo, além de conter as estratégias adotadas para efetivamente responder as perguntas de pesquisa.

\section{Definição} 

Yin (2001) define o estudo de caso como uma investigação empírica que analisa um determinado fenômeno da atualidade em seu contexto real, ou seja, o pesquisador se insere no ambiente cotidiano onde o objeto de estudo está sendo executado. Runeson e Höst (2009) defendem que esse método se adequa muito bem a diversas pesquisas realizadas na engenharia de software, devido a necessidade de analisar fenômenos contemporâneos interligados, o que dificulta sua análise de forma isolada.

Também, de acordo com Yin (2001), o estudo de caso é flexível e iterativo, isso significa que a estrutura do estudo pode se adaptar no decorrer da pesquisa, pois o pesquisador ao realizar as iterações de coleta e análise dos dados, pode vir a perceber características do caso que não foram possíveis serem identificar a priori.

Os estudos de caso devem coletar dados de múltiplas fontes, dessa forma, ao verificar que múltiplas fontes de dados apontam para a mesma conclusão, aumenta-se robustez e confiabilidade dos resultados, pois é diminuido a probabilidade de erro ou viés, além de fornecer uma visão mais ampla sobre o caso.

\section{Objetivo}

Segundo \cite{Siavvas2021431}, o desenvolvimento de software seguro é pautado na medição da qualidade da segurança de software, pois assim, é possível avaliar o nível da segurança do produto e conseguir traçar metas para guiar os processos de melhoria contínua do sistema. Porém, por diversas vezes são utilizados critérios de avaliação subjetivos ou que não possuem a devida validação, o que pode acarretar catastofrés relacionadas a segurança do produto.
Situação essa que se agrava ao se tratar de práticas emergentes na indústria, como DevSecOps, que apesar de seu destaque no desenvolvimento ágil por muitas vezes carece de avaliação por metodologias apropriadas.

Assim, o objetivo desse estudo consiste em descorbrir como as práticas DevSecOps afetam os aspectos relacionados a segurança de um projeto de software, usando métodos propostos por pesquisadores que compõe o estado da arte da engenharia de software para análise da segurança.

\section{Caso}

O MEPA - Contratos de Energia é um sistema web open-source criado pelo Laboratório Avançado de Produção, Pesquisa e Inovação em Software (LAPPIS) da Universidade de Brasília (UnB) e que recebe contribuições de alunos durante o semestre, devido à sua integração com a disciplina de Gerência de Configuração e Evolução de Software, também ministrada na UnB.


A arquitetura consiste em um frontend construído com o framework Next.js do React, que forma a interface com o usuário. Essa interface se comunica com um servidor Python construído com o framework Django. Os dados são persistidos em um banco de dados PostgreSQL que se conecta ao sistema pelo Django. A comunicação entre a interface e o servidor é feita por uma API REST; dessa forma, a interface gráfica se comunica apenas com o servidor, e este faz a comunicação com o banco de dados, fornecendo os dados necessários para a exibição na interface. Essa arquitetura pode ser observada na Figura 10.


\begin{figure}[h]
    \centering
	\caption{Arquitera Geral}
    \includegraphics[width=0.9\textwidth]{figuras/arquitetura-geral.png}

    \label{fig:arquitetura-geral}
	Fonte: LAPPIS
\end{figure}


A API (servidor) é composta por quatro módulos. O primeiro, chamado de Models, contém as definições dos objetos que serão armazenados no banco de dados, incluindo seus atributos e os comportamentos básicos de criar, remover, atualizar e deletar.

O módulo Serializer é responsável por serializar e desserializar os objetos definidos nos Models. Ele traduz e valida os dados entre o formato de comunicação da API (JSON) e o formato mais complexo usado internamente pelo framework.

Após a tradução dos objetos, o módulo de visualização (Views) realiza o processamento das requisições e respostas HTTP, que são capturadas pelo módulo de roteamento de requisições. Essa arquitetura está ilustrada na Figura 11.

\begin{figure}[h]
    \centering
	\caption{Arquitera do Backend}
    \includegraphics[width=0.9\textwidth]{figuras/arquitetura-backend.png}

    \label{fig:arquitetura-backend}
	Fonte: LAPPIS
\end{figure}


A Figura 12 representa a arquitetura da interface gráfica (frontend), construída com o framework Next.js e a biblioteca React para a criação de componentes. Utiliza-se também a biblioteca Redux para gerenciar os estados e manter a consistência do fluxo de dados da aplicação. Os componentes são a parte principal da interface, pois é através deles que o usuário interage com o sistema. Eles representam todos os elementos visuais e são utilizados para evitar o acoplamento do código e facilitar a resolução de problemas.

Quando o usuário executa uma ação em um componente, uma função criadora de ação é chamada. Essa função gera um objeto de ação que descreve o que aconteceu. Em seguida, essa ação é enviada para um redutor. O redutor, por sua vez, avalia a ação e determina como o estado da aplicação deve mudar, retornando um novo estado atualizado.

Esse novo estado é salvo na Store, que centraliza e armazena todas as informações de estado da aplicação. Quando a interface detecta uma mudança na Store, os novos dados são passados para os componentes relevantes por meio das Propriedades (Props), garantindo que a interface do usuário reflita o estado atual da aplicação.



\begin{figure}[h]
    \centering
	\caption{Arquitera do Backend}
    \includegraphics[width=0.9\textwidth]{figuras/arquitetura-frontend.png}

    \label{fig:arquitetura-frontend}
	Fonte: LAPPIS
\end{figure}

\section{Trabalhos Relacionados}


\section{Questão de Pesquisa}

\begin{itemize}
    \item Questão Específica 1: Qualidade Interna e Externa
    \begin{itemize}
        \item Métrica 1.1: SAST e SCA
        \item Métrica 1.2: DAST
    \end{itemize}
    \item Questão Específica 2: Análise da segurança
    \begin{itemize}
        \item Métrica 2.1: builds
        \item Métrica 2.2: ponto de vista do time
    \end{itemize}
\end{itemize}


\section{Fonte de Dados}

Código-fonte

Sistema em uso

Orquestrador de CI/CD

Percepção da equipe

\section{Procedimentos}

Integração e configuração das ferramentas de segurança ao pipeline de CI/CD

Análise de Baseline de Segurança (estado atual do projeto)

Execução Contínua da Análise de Segurança

Análise dos Dados

Coleta de Percepção da Equipe

\section{Análise de Dados}

Estatística Descritiva e Análise de Tendência

Análise de Frequência de Respostas

\section{Instrumentação}

SAST: SonarQube

SCA: Trivy

DAST: OWASP ZAP

Bloqueio de Build/Deploy: GitLab CI/CD

Questionário: Google Forms