\chapter[Planejamento do Estudo de Caso]{Planejamento do Estudo de Caso}
\label{chap:planejamento-estudo-caso}

Neste capítulo é apresentado o planejamento do estudo de caso, que será conduzido na segunda etapa do trabalho, onde são estabelecidos os conceitos e definições relacionados a esse método. 

\section{Definição} 

\citeonline{yin2015estudo} define o estudo de caso como uma investigação empírica que analisa um determinado fenômeno da atualidade em seu contexto real, ou seja, o pesquisador se insere no ambiente cotidiano onde o objeto de estudo está sendo executado. Essa abordagem é particularmente relevante para a Engenharia de Software, pois frequentemente os fenômenos analisados nessa área são complexos e interligados, o que dificulta analisá-los isoladamente do seu ambiente de execução \cite{Runeson2009}.

Ainda de acordo com \citeonline{yin2015estudo}, o estudo de caso é flexível e interativo. Isso significa que a estrutura do estudo pode se adaptar no decorrer da pesquisa, pois, o pesquisador, ao realizar as interações de coleta e análise dos dados, pode vir a perceber características do caso que não foram identificadas a priori.

Ademais, visando assegurar o rigor e a confiabilidade da investigação, os estudos de caso devem coletar dados de múltiplas fontes. Dessa forma, ao verificar que conclusões obtidas através de dados obtidos em diferentes locais apontam para o mesmo resultado, aumenta-se a robustez e confiabilidade dos resultados, pois esse processo de triangulação diminui a probabilidade de erro ou viés, além de fornecer uma visão mais ampla sobre o caso \cite{yin2015estudo}.

Também é necessário determinar se o estudo será holístico ou incorporado \cite{yin2015estudo}. Os estudos de caso holísticos analisam o caso como um todo, ou seja, o fenômeno é visto como um sistema único e integrado, portanto, o pesquisador obtém uma visão ampla e geral do caso. Já os estudos incorporados, tem como principal característica o aprofundamento em múltiplas unidades de análise dentro de um mesmo caso, permitindo uma análise mais aprofundada.

Para além disso, \cite{Runeson2009} explicita a necessidade de considerar as ameaças a validade do estudo desde o início da pesquisa, pois ignorar tais fatores interfere na confiabilidade dos resultados. A primeira ameaça definida pelo autor está relacionada à validade do constructo, que representa em qual extensão as medidas operacionais realmente representam o que o pesquisador tem em mente e o que está sendo investigado de acordo com as perguntas de pesquisa. Já a ameaça a validade interna, refere-se ao risco de existir um fator não mapeado pelo pesquisador que cause interferência na relação causal entre os fatores selecionados.
As ameaças externas, por outro lado, estão relacionadas a capacidade de generalização do estudo, ou seja, deseja-se que os achados do estudo tenham relevância para casos com características semelhantes. Por fim, a confiabilidade refere-se ao grau em que os dados e a análise dependem de pesquisadores específicos, isto é, se outro pesquisador conduzisse o mesmo estudo, o resultado deveria ser o mesmo.

\section{Objetivo}

Segundo \citeonline{Siavvas2021431}, o desenvolvimento de software seguro é pautado na medição da qualidade da segurança de software, pois assim, é possível avaliar o nível da segurança do produto e conseguir traçar metas para guiar os processos de melhoria contínua do sistema. Porém, por diversas vezes são utilizados critérios de avaliação subjetivos ou que não possuem a devida validação, o que pode acarretar catastofrés relacionadas a segurança do produto.
Situação essa que se agrava ao se tratar de práticas emergentes na indústria, como DevSecOps, que apesar de seu destaque no desenvolvimento ágil, por muitas vezes, carece de avaliação por metodologias apropriadas.

Assim, o objetivo desse estudo consiste em investigar como as práticas DevSecOps afetam os aspectos relacionados a segurança de um projeto de software livre em desenvolvimento. Para tanto serão utilizados métodos, técnicas e ferramentas, observadas e propostos na literatura específica da área de segurança e em consonância com o estado da prática na indústria.

\section{Caso}
O caso em estudo neste trabalho é a plataforma Brasil Participativo \footnote{https://brasilparticipativo.presidencia.gov.br/}. Trata-se de uma plataforma digital de participação social, que contribui para a criação, monitoramento e  aperfeiçoamento de políticas públicas brasileiras.  A plataforma é de de responsabilidade da Secretaria Nacional de Participação Social da Secretaria Geral da Presidência da República (SNPS/SGPR)\footnote{https://www.gov.br/secretariageral/pt-br/composicao/orgaos-especificos-singulares/snps}. Essa plataforma é um produto de software livre, disponibilizado nos termos da licença GNU AGPL-3.0, e foi desenvolvida com o apoio da Dataprev\footnote{https://www.dataprev.gov.br/}, da comunidade Decidim-Brasil\footnote{https://decidim.org/pt-br/}, do Ministério da Gestão e da Inovação em Serviços Públicos (MGI)\footnote{https://www.gov.br/gestao/pt-br}  e da Universidade de Brasília (UnB), por meio do Laboratório Avançado de Produção, Pesquisa e Inovação em Software(LAPP1S)\footnote{https://www.lappis.rocks/}. 
O Brasil Participativo tem como objetivo promover a interação entre cidadãos e o governo, permitindo que os cidadãos e cidadãs,  participem de consultas públicas, conferências, planos e enquetes sobre políticas públicas. 
A versão em uso do produto fica hospedada na infraestrutura computacional da Dataprev\footnote{https://www.dataprev.gov.br/}. O Brasil participativo tem como base o produto de software mundialmente utilizado, o Decidim\footnote{https://decidim.org/}   \cite{BrasilParticipativo}.

O Brasil Participativo possibilitou que 1.619.015 pessoas de diversas regiões do país contribuíssem com ideias de impacto nacional de forma simples, segura e transparente. Em acesso ao site da plataforma, no dia 10/11/2025, constava a informação de aproximadamente 1.619.015 participantes, 9.200.013 acessos e um total de 47 processos desde o seu laçamento. Esses números evidenciam seu papel no fortalecimento e apoio à democracia brasileira, ao promover a diversidade e a inclusão na formulação de políticas públicas \cite{BrasilParticipativo}.

O Decidim é um produto de software livre disponibilizado sob a licença GNU AGPL-3.0. Trata-se de um framework democrático, participativo, genérico, baseado em Ruby on Rails. Ele foi desenvolvido pelo governo de Barcelona para promover a participação cidadã e a democracia participativa. Ele permite que qualquer organização faça uma adaptação de seus componentes para a própia realidade, desse modo, tornou-se uma referência em tecnologia cívica. \cite{Decidim}

O Brasil Participativo \footnote{https://gitlab.com/lappis-unb/decidimbr/decidim-govbr} é desenvolvido por várias equipes multidisciplinares, compostas principalmente pelos perfis de  jornalistas, redatores publicitários, designers gráficos, desenvolvedores, especialistas em segurança da informação, especialistas em infraestrutura, gerentes de projeto e líderes técnicos. Cada um dos times tem um líder responsável por coordenar as atividades e garantir a qualidade do trabalho realizado. Em acesso ao repositório do código-fonte da plataforma, no dia 10/11/2025, haviam 4.192 commits\footnote{https://gitlab.com/lappis-unb/decidimbr/decidim-govbr/-/commits/main} e 103 versões de produto(releases)\footnote{https://gitlab.com/lappis-unb/decidimbr/decidim-govbr/-/tags?page=5} disponibilizadas.

 A  \ref{fig:organograma-equipes} apresenta o organograma das equipes envolvidas no projeto. A equipe de desenvolvimento é composta por 9 desenvolvedores, enquanto a equipe de segurança conta com 5 especialistas dedicados a garantir a integridade e proteção do sistema.

\begin{figure}[h]
    \centering
	\caption{Organograma das equipes do projeto Brasil Participativo}
    \includegraphics[width=1.0\textwidth]{figuras/organograma-equipes.png}

    \label{fig:organograma-equipes}
	Fonte: Autor
\end{figure}

Já a Figura \ref{fig:bpmn} apresenta uma visão em alto nível, do processo de desenvolvimento do Brasil Participativo. Estão representadas na figura as atividades envolvidas no processo de melhoria contínua do projeto, dado que esse é o foco deste trabalho. O início do fluxo se dá com a definição dos novos requisitos a serem implementados, sendo que os mesmos são escritos como histórias de usuário. Posteriosmente, essas histórias são documentadas e repassadas para o time de desenvolvimento, que é responsável por implementar as mudanças no código-fonte. Após a implementação, o código é submetido a uma série de testes automatizados e manuais para garantir sua qualidade e segurança. Após os testes serem executados, o time de segurança divulga os resultados para o time de desenvolvimento, que é responsável por corrigir as vulnerabilidades encontradas. Por fim, o código é  integrado ao sistema principal pelo time de infraestrutura, concluindo o ciclo. 

Este estudo de caso concentra-se nas etapas posteriores ao desenvolvimento e anteriores à entrada em produção. Nesse estágio, ferramentas de segurança são executadas por meio da pipeline de CI/CD para realizar a análise estática do código-fonte, bem como a análise dinâmica, que depende da implantação(deploy), pois considera aspectos do comportamento em uso do produto.

\begin{figure}[h]
    \centering
	\caption{BPMN do processo de desenvolvimento do Brasil Participativo}
    \includegraphics[width=1.0\textwidth]{figuras/bpmn.png}

    \label{fig:bpmn}
	Fonte: Autor
\end{figure}

Esse projeto foi escolhido por conveniência da observação do fenômeno. Trata-se de um poduto de software livre que conta com equipes de segurança e desenvolvimento acessíveis e ativas, o que facilita a comunicação e a colaboração durante o estudo de caso. Além disso, o LAPP1S possui um ambiente de desenvolvimento ágil, cultura DevSecOps e entregas contínuas, o que é fundamental para a avaliação das práticas implementadas, além de o software ser de grande relevância social, o que reforça a importância de garantir sua segurança e confiabilidade.

Porém, apesar de já possuir o Brakeman, uma ferramenta SAST, implementada em sua pipeline de CI/CD, foi identificado que ela só analisa três diretórios em todo o projeto. Nesse sentido, foi fundamental a conversa com o time de segurança para mudar a abordagem de análise da ferramenta para o contexto do projeto como um todo, o que apesar de representar uma mudança na cultura do projeto, é fundamental para uma análise quantitativa de fato representativa. 

Ademais, outro aspecto fundamental para a realização do estudo foi optar por marcar a pipeline como sinalizada ao encontrar uma vulnerabilidade, ou seja, a pipeline executa até o fim mesmo se encontrar problemas de segurança, porém fica marcada com uma sinalização de aviso. Essa decisão foi tomada, pois ao implementar novas ferramentas que analisam pontos até então não abarcados por testes de segurança é esperado que muitas vulnerabilidades sejam descobertas, e levando em consideração o volume de trabalho que o time está submetido, falhar a pipeline significaria bloquear a produtividade nas demandas atuais.

\section{Trabalhos Relacionados}

Alguns trabalhos obtidos por meio da revisão da literatura foram utilizados como referência para a realização desta pesquisa. Esses artigos foram selecionados, pois abordam os temas pertinentes e correlatos, para o planejamento e execução da pesquisa deste estudo de caso. 

O estudo realizado por \citeonline{Siavvas2021431} foi o estudo central para a elaboração da monografia. Sua relevância se dá, pois ele apresenta um modelo hierárquico de avaliação de segurança que quantifica a segurança interna do software com base em alertas de análise estática (SAST) e métricas de software.
O autor demonstrou que é possível avaliar a segurança de forma confiável e quantiativa usando modelos de qualidade e análise estática.

O trabalho de \cite{Zhang2024160317}, apresenta doze métricas quantitativas de DevSecOps especificamente projetadas para microsserviços web baseados em nuvem. O foco é avaliar a qualidade, segurança e eficiência operacional, com o intuito de auxiliar na tomada de decisões informadas e na melhoria contínua.

A revisão sistemática feita por \citeonline{Rajapakse2022}, identifica os desafios e soluções na adoção do DevSecOps, incluindo seleção de ferramentas, avaliação contínua de segurança e o equilíbrio entre velocidade e segurança.

\citeonline{Saeed2025139} aborda técnicas para integrar a segurança no ciclo de desenvolvimento de software, enfatizando a necessidade de uma abordagem colaborativa e ferramentas automatizadas para análise de ameaças e testes de segurança.

\section{Questão de Pesquisa}
\label{sec:questoes-caso}

De forma análoga ao procedimento adotado no planejamento da revisão da literatura, empregou-se a abordagem GQM \citeonline{Basili1994} para a definição das questões específicas, derivadas da questão principal, bem como das métricas correspondentes. Essa abordagem visa assegurar a aderência ao objetivo central do estudo e possibilitar a avaliação quantitativa ou qualitativa de cada questão de pesquisa secundária, que atualmente orienta a condução do estudo de caso.

A \citeonline{ISO25010} define a confidencialidade, como a capacidade do sistema de impedir o acesso não autorizado às informações, assim, impedindo que os dados privados sejam visíveis para quem não possui as permissões necessárias. Ela é uma subcaracterística da característica de segurança e será analisada neste estudo de caso, por meio da análise de vulnerabilidades detectadas no pipeline de CI/CD. Para corroborar com essa análise e fazer a triangularização da coleta de dados, uma avaliação qualitativa com os membros do time é necessária para observar os impactos dessas novas práticas no processo de desenvolvimento.

À vista disso, foram elaboradas as seguintes questões específicas deste estudo:

\begin{itemize}
    \item Questão Específica 1 (QE-1): A aplicação de práticas DevSecOps permitiu identificar vulnerabilidades de segurança sob as perspectivas da qualidade interna e externa do produto?
    \begin{itemize}
        \item Métrica 1.1: Média de vulnerabilidades encontradas por ferramentas SAST, por nível de severidade, por merge \cite{ISO27004}.
        \item Métrica 1.2: Média de vulnerabilidades encontradas por ferramentas DAST, por nível de severidade, por merge \cite{ISO27004}.
        \item Métrica 1.3: Proporção de vulnerabilidades em bibliotecas encontradas por ferramentas SCA, por nível de severidade, por merge \cite{Zhang2024160317}.
        \item Métrica 1.4: Security Incidents Trend pelas ferramentas SAST e DAST, por nível de severidade, por merge \cite{ISO27004}.
    \end{itemize}
    \item Questão Específica 2 (QE-2): A análise automática da segurança do pipeline e as métricas coletadas ajudaram na tomada de decisões relacionadas ao projeto?
    \begin{itemize}
        \item Métrica 2.1: Change Fail Rate, por merge \cite{DORA2024}.
        \item Métrica 2.2: Opinião do time obtido por um questionário sobre os impactos das novas práticas.
    \end{itemize}
\end{itemize}


\section{Fonte de Dados}

Para coletar os dados necessários para a posterior avaliação das métricas são necessárias diferentes fontes de dados. Primeiramente, as ferramentas SAST e SCA são executadas diretamente no código-fonte. 
Outra fonte de dados é o sistema em uso, que será usado para a obtençao dos dados referentes às ferramentas DAST que analisam o software em execução.
Adiconalmente, o orquestrador de CI/CD atuará como fonte de dados para coletar os tentativas falhas de integração do código, evidênciando a identificação de falhas pelas ferramentas.

Por fim, a equipe técnica será a fonte de dados dos formulários de avaliação ao final do estudo de caso, permitindo a análise qualitativa e triangularização dos resultados.

\section{Unidades de Análise e Procedimentos}

Para atender à complexidade das questões de pesquisa deste trabalho, optou-se pelo método de estudo de caso incorporado ou embutido, fundamentado por \citeonline{yin2015estudo}. 
Esta abordagem foi escolhida, pois permite que o caso principal seja investigado por meio da análise aprofundada de múltiplas subunidades de análise. Especificamente, o estudo se debruça sobre diferentes faces do projeto, como
o impacto na qualidade da segurança interna e externa do produto, e a percepção da equipe sobre as mudanças implementadas. Ao examinar cada um desses componentes seguindo o protocolo de estudo de caso, é possível construir uma visão detalhada e triangulada que
fortalece a validade das conclusões e oferece uma resposta mais completa ao problema investigado.

A unidade de análise primária envolve a avaliação quantitativa da segurança do software. As métricas de segurança serão coletadas automaticamente nos Merges Requests (MRs) integrados na branch principal (main) do projeto. Dessa forma, será possível monitorar a evolução da segurança do software ao longo do tempo, permitindo a identificação de tendências e padrões relacionados à detecção de vulnerabilidades. Os MRs foram adotados como unidade de análise, pois eles fornecem a granularidade ideal para o tempo previsto para o estudo ao mesmo tempo que conseguem transmitir a situação das principais alterações no código-fonte. 
O ideal seria ter a release como a unidade de análise. Contudo, durante o período de execução desse estudo, não foi gerada uma quantidade de releases que figurasse como uma amostra estatística significativa. Por isso, optou-se por um grão menor da versão do produto, no caso, as versões geradas após a revisão e aceite do código-fonte, no fechamento de um determinado MR.

A outra unidade de análise envolve a equipe técnica responsável pelo desenvolvimento e manutenção do software, cuja percepção sobre as mudanças será avaliada por meio de um questionário. O objetivo é coletar a opinião da equipe sobre o impacto das práticas DevSecOps no processo de desenvolvimento, identificando benefícios, desafios e sugestões de melhoria. Esta unidade de análise fornecerá dados qualitativos que complementarão as análises quantitativas das outras unidades.

\section{Análise de Dados}

A análise dos dados quantitativos será fundamentada em estatísticas descritivas e em análise de tendência, visando avaliar a evolução da segurança da aplicação ao longo do tempo. Para as métricas de detecção de vulnerabilidades por ferramentas SAST e DAST, Métrica 1.1 e Métrica 1.2, referentes a métrica B.34 - Security Incidents Trend da norma \citeonline{ISO27004}, será calculado um indicador de tendência para permitir avaliar o volume de vulnerabilidades de criticidade média ou superior, por nível de severidade.

O cálculo do indicador é obtido através da razão entre a média de vulnerabilidades das últimas duas sprints concluídas e a média das últimas seis sprints concluídas, sendo que, um valor inferior a 1.0 indica uma tendência de melhoria, entre 1.0 e 1.3 sinaliza uma tendência de estabilidade e superior a 1.3 indica uma piora significativa.

Para a análise de tendência, definimos a média recente ($M_{rec}$) e a média histórica ($M_{hist}$) como:

\begin{equation}
    M_{rec} = \frac{1}{2} \sum_{i=n-1}^{n} x_i
\end{equation}

\begin{equation}
    M_{hist} = \frac{1}{6} \sum_{i=n-5}^{n} x_i
\end{equation}

Onde $x_i$ representa o valor da métrica no intervalo de tempo $i$ e $n$ é o intervalo atual. A razão de tendência ($R$) é dada por:

\begin{equation}
    R = \frac{M_{rec}}{M_{hist}}
\end{equation}

Para a Métrica 1.3, adaptada da métrica Shared or Unknown Library Ratio proposta por \citeonline{Zhang2024160317}, será calculada a proporção de bibliotecas com vulnerabilidades em relação ao total. \citeonline{Zhang2024160317} argumento que o uso de bibliotecas de terceiros aumenta o vetor de ataque e dependência de terceiros no código da aplicação. Porém, o foco desse estudo é avaliar a segurança do software em si, portanto, a métrica será adaptada para considerar apenas as bibliotecas que apresentam vulnerabilidades de criticidade média ou superior. A métrica será calculada conforme a Equação \ref{eq:metrica-1-3}:
\begin{equation}
\label{eq:metrica-1-3}
    P_{vuln} = \frac{N_{vuln\_libs}}{N_{total\_libs}} \times 100
\end{equation}

Por sua vez, a Métrica 2.1, uma adaptação da métrica Change Fail Rate elaborado pelo \citeonline{DORA2024}, será calculada pela taxa percentual de merges sinalizados por vulnerabilidades de criticidade média ou superior. Se trata de uma adaptação, pois a métrica original considera falhas em produção, o que não é o foco deste estudo. A métrica será calculada conforme a Equação \ref{eq:metrica-2-1}:
\begin{equation}
\label{eq:metrica-2-1}
    CFR = \frac{N_{merges\_sinalizados}}{N_{total\_merges}} \times 100
\end{equation}

Para a análise de frequência de respostas do questionário, Métrica 2.2, a frequência de cada das respostas de cada pergunta será registrada, de modo a possibilitar o cálculo da moda, pois ao ter a opinião da maioria dos participantes sobre o tópico solicitado será possível obter a percepção geral do impacto das atividades realizadas.

\section{Instrumentação}

A instrumentação se refere às ferramentas que serão utilizadas para a realização do estudo de caso, aqui estão definidas as ferramentas usadas para a análise da segurança do sistema, controle de implementação do código, versionamento do código da pipeline feita e coleta de informações da equipe.

O Brakeman \footnote{\href{https://brakemanscanner.org/}{https://brakemanscanner.org/}} é uma ferramenta de open-source de avalição da qualidade e segurança do código-fonte. Ele realiza análise estática para detectar bugs, vulnerabilidades e code smells em várias linguagens. Ele fará parte das ferramentas white-box integradas ao pipeline realizando a análise estática de segurança de aplicação (SAST). Sua escolha deu-se em função da equipe já possuir experiência prévia com a ferramenta.

O Trivy \footnote{\href{https://trivy.dev/latest/}{https://trivy.dev/latest/}}
 é um scanner de segurança de código aberto. Ele é utilizado na análise de composição de software, verificando as bibliotecas de terceiros do projeto, garantindo que componentes externos ao projeto não insiram vulnerabilidades no sistemas. Além disso, ele é capaz de buscar vulnerabilidades em containeres e configurações de infraestrutura como código. Ele complementará o SonarQube como ferramenta white-box.

O Zed Attach Proxy-ZAP \footnote{\href{https://www.zaproxy.org/}{https://www.zaproxy.org/}} foi desenvolvido para encontrar problemas de segurança em aplicações web em execução. Ele será empregado para realizar a Análise Dinâmica de Segurança de Aplicação (DAST) em um ambiente de testes.

O GitLab CI/CD \footnote{\href{https://docs.gitlab.com/install/}{https://docs.gitlab.com/install/}} é uma ferramenta integrada ao GitLab que permite a automação das etapas do ciclo de vida do software, através dele que as ferramentas de segurança serão executadas automaticamente e ele servirá para bloquear o build/deploy caso vulnerabilidades sejam encontradas.

O Git \footnote{\href{https://git-scm.com/downloads}{https://git-scm.com/downloads}} é um sistema de controle de versão e o GitHub \footnote{\href{https://github.com/}{https://github.com/}} é uma plataforma de hospedagem de código-fonte para controle de versão com Git. Eles serão utilizados para o versionamento e armazenamento dos artefatos por este estudo de caso.

O Google Forms\footnote{\href{https://docs.google.com/forms/u/0/}{https://docs.google.com/forms/u/0/}} permite a criação rápida e fácil de formulários online, além de permitir a gestão e análise dos resultados. Ele será utilizado para a aplicação do questionário que coleta os dados qualitativos da equipe.
